\documentclass[answers,11pt]{exam}

\makeatletter
\newcommand{\course}[1]{\def\@course{#1}}
\makeatother

\makeatletter
\pagestyle{headandfoot}
\firstpageheader{}{\large \textbf{\@title \ifprintanswers ~-- Solutions \else \fi} \\ \@course}{}
\makeatother



\usepackage{amsmath}
\usepackage{amssymb}
\usepackage{booktabs}
\usepackage{graphicx}
\usepackage{subfig}

\DeclareGraphicsExtensions{.png,.pdf}


\DeclareMathOperator*{\Prob}{P}
\renewcommand{\Pr}{\Prob}
\DeclareMathOperator*{\E}{E}
\DeclareMathOperator*{\var}{var}
\DeclareMathOperator*{\sd}{sd}


\title{Counting}
\course{COR1-GB.1305 -- Statistics and Data Analysis}

\begin{document}

\begin{questions}


\fullwidth{\section*{The Multiplication Rule}}

\question A man has 4 pair of pants, 6 shirts, 8 pairs of socks, and 3 pairs of
shoes.  Ignoring the fact that some of the combinations may look ridiculous,
how many ways can he get dressed?

\begin{solution}
Using the multiplication rule, there are
\[
  4 \cdot 6 \cdot 8 \cdot 3 = 576
\]
ways for the man to get dressed.
\end{solution}

\vspace{\stretch{1}}


\question A restaurant offers soup or salad to start, and has 11 entr\'ees to choose
from, each of which is served with rice, baked potato, or zucchini.   How many
meals can you have if you can choose to eat one of their 4 desserts or have no
desert?

\begin{solution}
\[
  2 \cdot 11 \cdot 3 \cdot 5 = 330
\]

Note that there are $5$ choices for the final course (4 desserts or no
dessert).

\end{solution}

\vspace{\stretch{1}}


\question How many answer sheets are possible for a true/false test with 15
questions?

\begin{solution}
\[
  2^{15} = 32768
\]
\end{solution}

\vspace{\stretch{1}}



\fullwidth{\section*{Permutations}}


\question How many ways can 5 people stand in line?

\begin{solution}
\[
  5 \cdot 4 \cdot 3 \cdot 2 \cdot 1 = 5! = 120
\]
\end{solution}

\vspace{\stretch{1}}


\question How many different batting orders are possible for 9 baseball players?

\begin{solution}
\[
  9! = 362880
\]
\end{solution}

\vspace{\stretch{1}}


\question How many ways can 8 books be put on a shelf?

\begin{solution}
\[
  8! = 40320
\]
\end{solution}

\vspace{\stretch{1}}


\newpage

\fullwidth{\section*{More Permutations}}

\question Twelve people belong to a club.  How many ways can they pick a
president, vice-president, secretary, and treasurer?

\begin{solution}
\[
  12 \cdot 11 \cdot 10 \cdot 9 = \frac{12!}{8!} = 11880
\]
\end{solution}

\vspace{\stretch{1}}


\question In a horse race the first three finishers are said to win, place, and
show.  How many finishes are possible for a race with 11 horses?

\begin{solution}
\[
  11 \cdot 10 \cdot 9 = \frac{11!}{8!} = 990
\]
\end{solution}

\vspace{\stretch{1}}


\question Five different awards are to be given to a class of 30 students.  How
many ways can this be done if (a) each student can receive any number of
awards, (b) each sutent can receive at most one award?

\begin{solution}
(a) $30^5 = 24300000$ \\
(b) $30!/(25!) = 30 \cdot 29 \cdot 28 \cdot 27 \cdot 26 = 17100720$
\end{solution}

\vspace{\stretch{1}}



\fullwidth{\section*{Combinations}}

\question A club has 23 members.  


\begin{parts}

\part How many ways can they pick 2 people to be on a
committee to plan a party?

\begin{solution}
\[
  \binom{23}{2} = \frac{23 \cdot 22}{2 \cdot 1} = 253.
\]
\end{solution}

\vspace{\stretch{1}}

\part How many ways can they pick 4 people to be on a
committee to plan a party?

\begin{solution}
\[
  \binom{23}{4} = \frac{23 \cdot 22 \cdot 21 \cdot 20}{4 \cdot 3 \cdot 2 \cdot 1} = 8855.
\]
\end{solution}

\vspace{\stretch{1}}

\end{parts}


\question A restaurant offers 15 possible toppings for its pizza.  How many
different pizzas with 3 toppings can be ordered?

\begin{solution}
\[
  \binom{15}{3} = \frac{15 \cdot 14 \cdot 13}{3 \cdot 2 \cdot 1} = 455
\]
\end{solution}

\vspace{\stretch{1}}


\question We are going to pick 5 cards out of a deck of 52.  In how many ways
can this be done?

\begin{solution}
\[
  \binom{52}{5}
  = \frac{52 \cdot 51 \cdot 50 \cdot 49 \cdot 48}{5 \cdot 4 \cdot 3 \cdot 2 \cdot 1}
  = 2598960.
\]
\end{solution}


\vspace{\stretch{1}}

\newpage

\fullwidth{\section*{Advanced Problems}}

%\question Suppose we pick 4 balls out of an urn with 12 red balls and 8 black
%balls.  What is the probability of $B = \text{``We get two balls of each
%color''}$?
%
%\begin{solution}
%The number of ways to pick 4 balls out of 20 is $\binom{20}{4}$.
%
%We use the multiplication rule as to count the number of ways to pick two
%balls of each color.  The sequence of experiments is (1) pick two red balls
%from the $12$, then (2) pick two black balls from the $8$.  There are
%$\binom{12}{2}$ ways to perform the first experiment, and $\binom{8}{2}$ to
%perform the second experiment, so there are $\binom{12}{2} \binom{8}{2}$
%total.
%
%Finally,
%\begin{align*}
%  P(B) &=
%    \frac{\#\{\text{picks with two balls of each color}\}}
%         {\#\{\text{picks with four balls\}}} \\
%       &= \frac{\binom{12}{2} \binom{8}{2}}{\binom{20}{4}}
%\end{align*}
%\end{solution}
%
%\vspace{\stretch{3}}
%
%\ifprintanswers\else\newpage\fi


\question \textbf{New York state lotto.}  You pick six of the numbers 1 through
54, and then in a televised drawing six of the numbers are selected.  If all
six of your numbers are selected then you win a share of the first place
prize.  If five or four of your numbers are selected you win a share of the
second or third prize.

\begin{parts}
\part How many ways are there to select 6 numbers for the lotto ticket?

\begin{solution}
\[
  \binom{54}{6}
  = \frac{54 \cdot 53 \cdot 52 \cdot 51 \cdot 50 \cdot 49}
         {6 \cdot 5 \cdot 4 \cdot 3 \cdot 2 \cdot 1}
  = 25827165
\]
\end{solution}

\vspace{\stretch{1}}


\part How many ways are there to select a first prize number?

\begin{solution}
\[
  1
\]
\end{solution}

\vspace{\stretch{1}}


\part What is the probability of selecting a first prize number?


\begin{solution}
\begin{align*}
   P(\text{first prize})
     &= \frac{\#\{\text{lotto tickets that match all six numbers}\}}
         {\#\{\text{lotto tickets\}}} \\
     &= \frac{1}{\binom{54}{6}} \\
     &= 1 / 25827165 \\
     &= 0.000004\%
\end{align*}
\end{solution}

\vspace{\stretch{1}}

\end{parts}


%How many lotto tickets have five matches and one mismatch from our pick?
%Consider the following sequence of experiments for enumerating all such
%possibilities: (1) pick $5$ numbers out of our six to match; (2) pick $1$
%number to mismatch.  There are $\binom{6}{5}$ possible outcomes for experiment
%(1), and $\binom{48}{1}$ possible outcomes for experiment (2).  (Note that $48
%= 54 - 6$; there are $48$ numbers in the range 1--54 which aren't on our lotto
%ticket.)  Thus,
%\begin{align*}
%   P(\text{second prize})
%    &= \frac{\#\{\text{lotto tickets that match exactly five numbers}\}}
%         {\#\{\text{lotto tickets\}}} \\
%     &= \frac{\binom{6}{5}\binom{48}{1}}{\binom{54}{6}}.
%\end{align*}
%
%The computation for third prize is similar to the computation for second
%prize:
%
%\begin{align*}
%   P(\text{third prize})
%    &= \frac{\#\{\text{lotto tickets that match exactly four numbers}\}}
%         {\#\{\text{lotto tickets\}}} \\
%     &= \frac{\binom{6}{4}\binom{48}{2}}{\binom{54}{6}}.
%\end{align*}


\vspace{\stretch{1}}


\question \textbf{Quality assurance.}  Suppose we have a batch of $100$ light bulbs,
which contains $5$ defective bulbs.  If we pick $10$ for testing, what is the
probability that no bulbs in the sample are defective?  We can answer this
question in three steps.

\begin{parts}

\part How many ways are there of picking $10$ bulbs for testing out of $100$?

\begin{solution}
\[
  \binom{100}{10}
\]
\end{solution}

\vspace{\stretch{1}}

\part How many ways are there of picking $10$ non-defective bulbs?

\begin{solution}
\[
  \binom{95}{10}
\]
\end{solution}

\vspace{\stretch{1}}

\part What is the probability that there are no defective bulbs in your sample
of $10$?

\begin{solution}
\begin{align*}
  P(\text{no defects in sample})
    &= \frac{\binom{95}{10}}{\binom{100}{10}} \\
    &= 58\%.
\end{align*}
\end{solution}

\vspace{\stretch{1}}

\end{parts}


%\ifprintanswers\else\newpage\fi

\end{questions}

\end{document}

