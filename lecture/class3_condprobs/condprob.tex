\documentclass[11pt]{exam}

\makeatletter
\newcommand{\course}[1]{\def\@course{#1}}
\makeatother

\makeatletter
\pagestyle{headandfoot}
\firstpageheader{}{\large \textbf{\@title \ifprintanswers ~-- Solutions \else \fi} \\ \@course}{}
\makeatother



\usepackage{amsmath}
\usepackage{amssymb}
\usepackage{booktabs}
\usepackage{graphicx}
\usepackage{subfig}
\usepackage{enumitem}

\DeclareGraphicsExtensions{.png,.pdf}


\DeclareMathOperator*{\Prob}{P}
\renewcommand{\Pr}{\Prob}
\DeclareMathOperator*{\E}{E}
\DeclareMathOperator*{\var}{var}
\DeclareMathOperator*{\sd}{sd}


\title{Conditional Probability and Counting Rules}
\course{STAT-UB.0001 -- Statistics for Business Control}

\begin{document}

\begin{questions}

\fullwidth{\section*{Multiplicative Rule}}
\question Suppose you run a lottery in the class of 20 students. You put  1 red ball and 19 green balls into a blackbox. One by one, each student randomly picks a ball from the blackbox, and whoever gets the red ball wins the lottery. 
\begin{parts}
\part What is the probability for the first student to win the lottery?
\begin{solution}
$$\frac{1}{20}$$
\end{solution}
\vspace{\stretch{1}}

\part What is the probability for the second student to win the lottery?
\begin{solution}
Let A=``the first student wins'', and B=``the second student wins''. Then,
\begin{align*}
P(B) &= P(A^c \cap B) \\
& = P(A^c) P(B\mid A^c)\\
&= \frac{19}{20} \frac{1}{19} \\
&= \frac{1}{20}
\end{align*}
\end{solution}
\vspace{\stretch{1}}

\part What is the probability for the last student to win the lottery?
\begin{solution}
Let 
\begin{align*}
A_1 &= \text{the first student wins}\\
A_2 &= \text{the second student wins}\\
\vdots &  \\
A_{20} &= \text{the 20-th student wins}
\end{align*}

Then,
\begin{align*}
P(A_{20}) &= P(A_1^c \cap A_2^c \cap \cdots \cap A_{19}^c \cap A_{20})\\
& = P(A_1^c) P(A_2^c \mid A_1^c) P(A_3^c \mid A_1^c \cap A_2^c)\\
& \cdots P(A_{19}^c \mid A_1^c \cap A_2^c \cap \cdots A_{18}^c)
P(A_{20} \mid A_1^c \cap A_2^c \cap \cdots A_{19}^c)\\
&= \frac{19}{20}  \frac{18}{19}  \frac{17}{18} \cdots \frac{1}{2} \\
&= \frac{1}{20}
\end{align*}

\end{solution}
\vspace{\stretch{1}}

\end{parts}

 \newpage
\fullwidth{\section*{Independence}}
\question Suppose that you flip two fair coins.  Let $A = $``the first coin shows
Heads,'' $B = $ ``The second coin shows Heads.''  Find the probability
of getting Heads on both coins, i.e.~find $P(A \cap B)$.

\begin{solution}
The long way to solve this problem is to write out the elementary outcomes and
their probabilities:

\vspace{1\baselineskip}
\begin{center}
\begin{tabular}{cc}
\toprule
Outcome & Probability \\
\midrule
HH & $\tfrac{1}{4}$ \\
HT & $\tfrac{1}{4}$ \\
TH & $\tfrac{1}{4}$ \\
TT & $\tfrac{1}{4}$ \\
\bottomrule
\end{tabular}
\end{center}
\vspace{1\baselineskip}

Since $A \cap B = \{ \text{HH} \}$, it follows that
\[
  \Pr(A \cap B) = \tfrac{1}{4}.
\]

We can solve this problem much more expediently using the independence of $A$
and $B$:
\begin{align*}
  \Pr(A \cap B)
  &= \Pr(A) \Pr(B \mid A) \\
  &= \Pr(A) \Pr(B) \\
  &= (\tfrac{1}{2}) (\tfrac{1}{2}) \\
  &= \tfrac{1}{4}.
\end{align*}

\end{solution}

\vspace{\stretch{1}}

\question Suppose that you roll two dice.  What is the probability of getting exactly
one 6?

\begin{solution}
Define the following events:
\begin{align*}
  A &= \text{``6 on the first roll,''} \\
  B &= \text{``6 on the second roll,''}
%  C &= \text{``6 on the third roll.''}
\end{align*}
Using the shorthand $\bar A = A^c$ and $A\bar B = A \cap \bar B$, the event
``exactly one 6'' can be written as
\[
  \text{``exactly one 6''}
  = A \bar B \cup \bar A B
\]
These events are mutually exclusive, so
\[
  \Pr(\text{exactly one 6})
  = \Pr(A \bar B) 
  + \Pr(\bar A B)
\]
Using the independence of events $A$ and $B$, we get
\begin{align*}
  \Pr(A \bar B)  &= \Pr(A) \Pr(\bar B) = (\tfrac{1}{6}) (\tfrac{5}{6}) \\
  \Pr(\bar A B)  &= \Pr(\bar A) \Pr(B) = (\tfrac{5}{6}) (\tfrac{1}{6})
\end{align*}
Note that these two expressions are equal.  Thus,
\[
  \Pr(\text{exactly one 6})
  = 2 \cdot \tfrac{1}{6} \cdot \tfrac{5}{6}
  \approx 28\%
\]


\end{solution}

\vspace{\stretch{1}}
%\input{indep-2dice-sum7-red6}
%\input{indep-2dice-sum8-red6}

\question Suppose that you sell fire insurance policies to two different
buildings in Manhattan, located in different neighborhoods.  You estimate that
the buildings have the following chances of being damaged by fire in the next
10 years: 5\%, and 1\%.  Assume that fire damages to the two buildings
are independent events.  Compute the probability that exactly one building
gets damaged by fire in the next 10 years.

\begin{solution}
  \[
    (.05) (.99) + (.95) (.01) = .059 = 5.9\%
  \]
\end{solution}
\vspace{\stretch{1}}

\newpage

\question Consider the following experiment. A hat contains two coins:
\begin{itemize}
  \item one coin, the ``fair'' coin, has 50\% chance of heads and 50\% chance of tails on
every flip;
  \item the other coin, the ``heads'' coin, has heads on both sides, so it
always lands heads on every flip.
\end{itemize}
You reach into the hat and pull out a
random coin, equally likely to get the fair coin or the heads coin. Then, you
flip this coin twice.

Define events $A$ and $B$ as
\begin{align*}
  A &= \text{the first flip lands heads} \\
  B &= \text{the second flip lands heads}.
\end{align*}

\begin{parts}

  \part Based on your intuition, do you think that $A$ and $B$ independent
    events?

  \vspace{\stretch{1}}

  \part \label{hat-coin-pa} Compute $P(A)$.

  \begin{solution}
    There are two possibilities with equal chances:
    either we pick the fair coin, or we pick the heads coin. We know that
    \[
      P(\text{fair coin}) = P(\text{heads coin}) = 0.5
    \]
    Given the coin, it is easy to compute the probabilities of heads:
    \begin{align*}
      P(\text{heads on first flip} \mid \text{fair coin})  &= 0.5 \\
      P(\text{heads on first flip} \mid \text{heads coin}) &= 1.0
    \end{align*}
    Finally,
    \begin{align*}
    P(A) &=
      P(\text{fair coin}) P(\text{heads on first} \mid \text{fair coin}) \\
      &\qquad+
      P(\text{heads coin}) P(\text{heads on first} \mid \text{heads coin}) \\
      &= (0.5) (0.5) + (0.5) (1.0) \\
      &= 0.75.
    \end{align*}

  \end{solution}

  \vspace{\stretch{1}}

  \part \label{hat-coin-pab} Compute $P(A \cap B)$.
    
  \begin{solution}
    Given the coin, the first and the second flips are independent:
    \begin{align*}
      P(\text{heads on both flips} \mid \text{fair coin})
        &= P(\text{heads on first flip} \mid \text{fair coin}) \\
        &\qquad\cdot
          P(\text{heads on second flip} \mid \text{fair coin}) \\
        &= (0.5) (0.5) \\
        &= 0.25.
    \end{align*}
    Similarly,
    \[
      P(\text{heads on both flips} \mid \text{heads coin}) = 1.0.
    \]
    Now,
    \begin{align*}
    P(A) &=
      P(\text{fair coin}) P(\text{heads on both} \mid \text{fair coin}) \\
      &\qquad+
      P(\text{heads coin}) P(\text{heads on both} \mid \text{heads coin}) \\
      &= (0.5) (0.25) + (0.5) (1.0) \\
      &= 0.625.
    \end{align*}
  \end{solution}

  \vspace{\stretch{1}}

  \part Use your answers to parts~(\ref{hat-coin-pa}) and~(\ref{hat-coin-pab})
    to either prove or disprove that $A$ and $B$ are independent.

  \begin{solution}
    To check for independence, we compare the product $P(A) P(B)$ with
    $P(A \cap B)$. Noting that $P(A) = P(B)$, we have
    \begin{align*}
      P(A) P(B) &= (0.75) (0.75) \\
                &= 0.5625.
    \end{align*}
    Clearly, $P(A) P(B) \neq P(A \cap B)$. Thus, the events are not
    independent.

    To get some more intuition for what is happening here, note that
    \begin{align*}
      P(B) &= 0.75, \\
      P(B \mid A) &= P(B \cap A) / P(A) \\
                  &= 0.625 / 0.75 \\
                  &= 0.833.
    \end{align*}
    That is, before performing the experiment, we have a 75\% chance of
    getting a heads on the second flip. In the middle of the experiment, if
    we see that the first flip is heads, then we have an 83.3\% chance of
    getting heads on the next flip. Why is this? After we see event $A$, we
    gain some information relevant to event $B$, namely that it is more
    likely we have selected the heads coin.
  \end{solution}

  \vspace{\stretch{1}}

\end{parts}

\newpage

\fullwidth{\section*{Bayes' Rule}}
\question  With probability 0.15, a person will pass the job interview for a Data Analyst position. 
Among those who passed the interview, 60\% had taken college courses in Statistics. It
happens also that 30\% of all those who interviewed had college courses in
Statistics. Find the probability that a person with college courses in Statistics
will pass the job interview.

%\question Every year in March there is a standardized exam for people who want to
%be licensed sheep herders. It happens that, with probability 0.45, a person
%will pass this exam. In the process of screening people, it turns out that
%among those who passed the exam, 60\% had taken college courses in biology. It
%happens also that 30\% of all those who take the exam had college courses in
%biology. Find the probability that a person with college courses in biology
%will pass the exam.

\begin{solution}
The information in the problem is
\begin{align*}
\Pr(\text{Pass}) &= .15 \\
\Pr(\text{Stats}) &= .30 \\
\Pr(\text{Stats} \mid \text{Pass}) &= .60
\end{align*}
The problem is asking us to compute the quantity
\(
  \Pr(\text{Pass} \mid \text{Stats}).
\)
Using Bayes' rule,
\begin{align*}
  \Pr(\text{Pass} \mid \text{Stats})
  &= \Pr(\text{Stats} \mid \text{Pass})
  \cdot \frac{\Pr(\text{Pass})}{\Pr(\text{Stats})}  \\
  &= (.60) \cdot \frac{(.15)}{(.30)} \\
  &= .30.
\end{align*}
That is, there is a 30\% chance that a person with college courses in statistics
will pass the exam.
\end{solution}

\vspace{\stretch{1}}

\question Amazon.com maintains a list of all registered customers, along with
their email addresses.  During July, they sent coupons to 20\% of their
customers.  They recorded that 5\% of their customers made purchases in July,
and 40\% of all purchases were made with coupons.  In this problem we will
compute the proportion of customers sent a coupon in July who made a
purchase in that month.  For simplicity, we will assume that customers either
make 0 or 1 purchases in July.

\begin{parts}
\part Consider a random customer, and define two events:
\begin{align*}
  \text{Coupon} &= \text{the customer received a coupon in July}, \\
  \text{Purchase} &= \text{the customer made a purchase in July}.
\end{align*}
Express all percentages given in the problem statement as
probabilities or conditional probabilities of these two events.
Example: $\Pr(\text{Coupon}) = 0.20$.


\begin{solution}
The information in the problem is
\begin{align*}
\Pr(\text{Coupon}) &= .20 \\
\Pr(\text{Purchase}) &= .05 \\
\Pr(\text{Coupon} \mid \text{Purchase}) &= .40
\end{align*}

\end{solution}

\vspace{\stretch{1}}

\part Use Bayes' rule to compute the proportion of custoers sent a
coupon in July who made a purchase that month.


\begin{solution}
The problem is asking us to compute the quantity
\(
  \Pr(\text{Purchase} \mid \text{Coupon}).
\)
Using Bayes' rule,
\begin{align*}
  \Pr(\text{Purchase} \mid \text{Coupon})
  &= \Pr(\text{Coupon} \mid \text{Purchase})
  \cdot \frac{\Pr(\text{Purchase})}{\Pr(\text{Coupon})}  \\
  &= (.40) \cdot \frac{(.05)}{(.20)} \\
  &= .10.
\end{align*}
\end{solution}

\vspace{\stretch{1}}

\end{parts}
\newpage
\question   Suppose that 1\% of population have a special disease. A blood test detects the disease with probability 0.95 when it is present, 
    but also falsely detects it when it's not present with probability 0.02. Test shows that a person has the disease; what is the probability that he indeed has it?
    \begin{solution}
The information in the problem is
\begin{align*}
\Pr(\text{Disease}) &= .01 \\
\Pr(\text{Diagnosed}\mid \text{Disease}) &= .95 \\
\Pr(\text{Diagnosed}\mid \text{not Disease}) &= .02 
\end{align*}

The problem is asking us to compute $\Pr(\text{Disease}\mid \text{Diagnosed})$. With Bayes' Rule,
\begin{align*}
& \Pr(\text{Disease}\mid \text{Diagnosed}) \\
&= \frac{\Pr(\text{Disease})\Pr(\text{Diagnosed}\mid \text{Disease})}
{\Pr(\text{Disease})\Pr(\text{Diagnosed}\mid \text{Disease})
+ \Pr(\text{not Disease})\Pr(\text{Diagnosed}\mid \text{not Disease})}\\
&= \frac{(.01)(.95) }{(.01)(.95)+(.99)(.02)} \\
&\approx 32.4\%
\end{align*}
\end{solution}
\vspace{\stretch{1}}

\question A desk lamp produced by a company was found to be defective (D). There are three factories (A, B, C) where such desk lamps are manufactured. A Quality Control Manager is responsible for investigating the source of found defects. This is what the manager knows about the company's desk lamp production and the possible source of defects: 

\begin{center}
\begin{tabular}{|c|c|c|} \hline
Factory & \% of total production & Probability of defective lamps \\ \hline
A &  0.35 & 0.015 \\ \hline
B &  0.35 & 0.010 \\ \hline
C &  0.30 & 0.020\\ \hline
\end{tabular}
\end{center}

If a randomly selected lamp is defective, what is the probability that the lamp was manufactured in factory C?
\begin{solution}
\begin{align*}
\Pr(C \mid D) & = \frac{\Pr(C \cap D)}{\Pr(D)} \\
& = \frac{\Pr(D)\Pr(C \mid D)}
{\Pr(A)\Pr(D\mid A) + \Pr(B)\Pr(D\mid B) + \Pr(C)\Pr(D \mid C)}\\
&= \frac{(0.30)(0.02)}{(0.35)(0.015)+(0.35)(0.01)+(0.30)(0.02)} \\
&= 0.407
\end{align*}
\end{solution}
\vspace{\stretch{1}}


\newpage
\fullwidth{\section*{The Multiplication Rule}}

\question A man has 4 pair of pants, 6 shirts, 8 pairs of socks, and 3 pairs of
shoes.  How many ways can he get dressed?

\begin{solution}
Using the multiplication rule, there are
\[
  4 \cdot 6 \cdot 8 \cdot 3 = 576
\]
ways for the man to get dressed.
\end{solution}

\vspace{\stretch{1}}


\question A restaurant offers soup or salad to start, and has 11 entr\'ees to choose
from, each of which is served with rice, baked potato, or zucchini.   How many
meals can you have if you can choose to eat one of their 4 desserts or have no
desert?

\begin{solution}
\[
  2 \cdot 11 \cdot 3 \cdot 5 = 330
\]

Note that there are $5$ choices for the final course (4 desserts or no
dessert).

\end{solution}

\vspace{\stretch{1}}


\question How many answer sheets are possible for a true/false test with 15
questions?

\begin{solution}
\[
  2^{15} = 32768
\]
\end{solution}

\vspace{\stretch{1}}



\fullwidth{\section*{Permutations}}


\question How many ways can 5 people stand in line?

\begin{solution}
\[
  5 \cdot 4 \cdot 3 \cdot 2 \cdot 1 = 5! = 120
\]
\end{solution}

\vspace{\stretch{1}}


\question How many different batting orders are possible for 9 baseball players?

\begin{solution}
\[
  9! = 362880
\]
\end{solution}

\vspace{\stretch{1}}


\question How many ways can 8 books be put on a shelf?

\begin{solution}
\[
  8! = 40320
\]
\end{solution}

\vspace{\stretch{1}}


\newpage

\fullwidth{\section*{More Permutations}}

\question Twelve people belong to a club.  How many ways can they pick a
president, vice-president, secretary, and treasurer?

\begin{solution}
\[
  12 \cdot 11 \cdot 10 \cdot 9 = \frac{12!}{8!} = 11880
\]
\end{solution}

\vspace{\stretch{1}}


\question In a horse race the first three finishers are said to win, place, and
show.  How many finishes are possible for a race with 11 horses?

\begin{solution}
\[
  11 \cdot 10 \cdot 9 = \frac{11!}{8!} = 990
\]
\end{solution}

\vspace{\stretch{1}}


\question Five different awards are to be given to a class of 30 students.  How
many ways can this be done if (a) each student can receive any number of
awards, (b) each sutent can receive at most one award?

\begin{solution}
(a) $30^5 = 24300000$ \\
(b) $30!/(25!) = 30 \cdot 29 \cdot 28 \cdot 27 \cdot 26 = 17100720$
\end{solution}

\vspace{\stretch{1}}



\fullwidth{\section*{Combinations}}

\question A club has 12 members.  


\begin{parts}

\part A club has 12 people. How many ways can they pick 2 people to be on a
committee to plan a party?

\begin{solution}
\[
  \binom{12}{2} = \frac{12 \cdot 11}{2 \cdot 1} = 66.
\]
\end{solution}

\vspace{\stretch{1}}

\part How many ways can they pick 4 people to be on a
committee to plan a party?

\begin{solution}
\[
  \binom{12}{4} = \frac{12 \cdot 11 \cdot 10 \cdot 9}{4 \cdot 3 \cdot 2 \cdot 1} = 495.
\]
\end{solution}

\vspace{\stretch{1}}

\end{parts}


\question A restaurant offers 15 possible toppings for its pizza.  How many
different pizzas with 3 toppings can be ordered?

\begin{solution}
\[
  \binom{15}{3} = \frac{15 \cdot 14 \cdot 13}{3 \cdot 2 \cdot 1} = 455
\]
\end{solution}

\vspace{\stretch{1}}


\question We are going to pick 5 cards out of a deck of 52.  In how many ways
can this be done?

\begin{solution}
\[
  \binom{52}{5}
  = \frac{52 \cdot 51 \cdot 50 \cdot 49 \cdot 48}{5 \cdot 4 \cdot 3 \cdot 2 \cdot 1}
  = 2598960.
\]
\end{solution}


\vspace{\stretch{1}}

\newpage

\fullwidth{\section*{Advanced Problems}}

%\question Suppose we pick 4 balls out of an urn with 12 red balls and 8 black
%balls.  What is the probability of $B = \text{``We get two balls of each
%color''}$?
%
%\begin{solution}
%The number of ways to pick 4 balls out of 20 is $\binom{20}{4}$.
%
%We use the multiplication rule as to count the number of ways to pick two
%balls of each color.  The sequence of experiments is (1) pick two red balls
%from the $12$, then (2) pick two black balls from the $8$.  There are
%$\binom{12}{2}$ ways to perform the first experiment, and $\binom{8}{2}$ to
%perform the second experiment, so there are $\binom{12}{2} \binom{8}{2}$
%total.
%
%Finally,
%\begin{align*}
%  P(B) &=
%    \frac{\#\{\text{picks with two balls of each color}\}}
%         {\#\{\text{picks with four balls\}}} \\
%       &= \frac{\binom{12}{2} \binom{8}{2}}{\binom{20}{4}}
%\end{align*}
%\end{solution}
%
%\vspace{\stretch{3}}
%
%\ifprintanswers\else\newpage\fi


\question \textbf{New York state lotto.}  You pick six of the numbers 1 through
54, and then in a televised drawing six of the numbers are selected.  If all
six of your numbers are selected then you win a share of the first place
prize.  If five or four of your numbers are selected you win a share of the
second or third prize.

\begin{parts}
\part How many ways are there to select 6 numbers for the lotto ticket?

\begin{solution}
\[
  \binom{54}{6}
  = \frac{54 \cdot 53 \cdot 52 \cdot 51 \cdot 50 \cdot 49}
         {6 \cdot 5 \cdot 4 \cdot 3 \cdot 2 \cdot 1}
  = 25827165
\]
\end{solution}

\vspace{\stretch{1}}


\part How many ways are there to select a first prize number?

\begin{solution}
\[
  1
\]
\end{solution}

\vspace{\stretch{1}}


\part What is the probability of selecting a first prize number?


\begin{solution}
\begin{align*}
   P(\text{first prize})
     &= \frac{\#\{\text{lotto tickets that match all six numbers}\}}
         {\#\{\text{lotto tickets\}}} \\
     &= \frac{1}{\binom{54}{6}} \\
     &= 1 / 25827165 \\
     &= 0.000004\%
\end{align*}
\end{solution}
\vspace{\stretch{1}}

\part What is the probability of selecting a second prize number?
\begin{solution}
How many lotto tickets have five matches and one mismatch from our pick?
Consider the following sequence of experiments for enumerating all such
possibilities: (1) pick $5$ numbers out of our six to match; (2) pick $1$
number to mismatch.  There are $\binom{6}{5}$ possible outcomes for experiment
(1), and $\binom{48}{1}$ possible outcomes for experiment (2).  (Note that $48
= 54 - 6$; there are $48$ numbers in the range 1--54 which aren't on our lotto
ticket.)  Thus,
\begin{align*}
   P(\text{second prize})
    &= \frac{\#\{\text{lotto tickets that match exactly five numbers}\}}
         {\#\{\text{lotto tickets\}}} \\
     &= \frac{\binom{6}{5}\binom{48}{1}}{\binom{54}{6}}.
\end{align*}

The computation for third prize is similar to the computation for second
prize:

\begin{align*}
   P(\text{third prize})
    &= \frac{\#\{\text{lotto tickets that match exactly four numbers}\}}
         {\#\{\text{lotto tickets\}}} \\
     &= \frac{\binom{6}{4}\binom{48}{2}}{\binom{54}{6}}.
\end{align*}
\end{solution}


\end{parts}





\vspace{\stretch{1}}
\newpage

\question \textbf{Quality assurance.}  Suppose we have a batch of $100$ light bulbs,
which contains $5$ defective bulbs.  If we pick $10$ for testing, what is the
probability that no bulbs in the sample are defective?  We can answer this
question in three steps.

\begin{parts}

\part How many ways are there of picking $10$ bulbs for testing out of $100$?

\begin{solution}
\[
  \binom{100}{10}
\]
\end{solution}

\vspace{\stretch{1}}

\part How many ways are there of picking $10$ non-defective bulbs?

\begin{solution}
\[
  \binom{95}{10}
\]
\end{solution}

\vspace{\stretch{1}}

\part What is the probability that there are no defective bulbs in your sample
of $10$?

\begin{solution}
\begin{align*}
  P(\text{no defects in sample})
    &= \frac{\binom{95}{10}}{\binom{100}{10}} \\
    &= 58\%.
\end{align*}
\end{solution}

\vspace{\stretch{1}}

\end{parts}


\newpage
\question \textbf{The Birthday Problem.} A class has 20 students.  What is the probability that at least
two students have the same birthday?  Assume that each person in the class was
assigned a random birthday between January 1 and December 31.

\begin{solution}
Assume that everyone in the class is randomly assigned a birthday, which
corresponds to number between 1 and 365 representing the day of the year.
It turns out to be much easier to compute the probability using the complement
rule, as
\[
  \Pr(\text{at least 2 people have the same birthday})
  = 1 - \Pr(\text{all 20 birthdays are different}).
\]

We use counting rules to compute the probability that all 20 birthdays are different. 
We first compute the size of sample space. That is $365^{20}$.

Next, we compute the number of ways to assign 365 birthdays to 20 students, such that no two students can have the same birthday.
That is $P(365,20)=\frac{365!}{345!}$.

Thus the probability of all 20 birthdays are different is
$$\frac{P(365,20)}{365^{20}} = \frac{365}{365}\cdot \frac{364}{365}\cdot\frac{363}{365}\cdots \frac{365-19}{365}.$$

\end{solution}





\end{questions}

\end{document}

