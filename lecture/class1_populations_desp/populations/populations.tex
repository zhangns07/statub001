\documentclass[11pt]{exam}

\makeatletter
\newcommand{\course}[1]{\def\@course{#1}}
\makeatother

\makeatletter
\pagestyle{headandfoot}
\firstpageheader{}{\large \textbf{\@title \ifprintanswers ~-- Solutions \else \fi} \\ \@course}{}
\makeatother



\usepackage{amsmath}
\usepackage{amssymb}
\usepackage{booktabs}
\usepackage{graphicx}
\usepackage{subfig}

\DeclareGraphicsExtensions{.png,.pdf}


\DeclareMathOperator*{\Prob}{P}
\renewcommand{\Pr}{\Prob}
\DeclareMathOperator*{\E}{E}
\DeclareMathOperator*{\var}{var}
\DeclareMathOperator*{\sd}{sd}


\title{Populations  and Samples}
\course{STAT-UB.0001 -- Statistics for Business Control}

\begin{document}

\begin{questions}

\fullwidth{%\section*{Populations and Bias}

Each of the following scenarios involves collecting data to learn about a
population.  State (a) what population is involved, and (b) why the sample is
biased.  To demonstrate that a sample is biased, you must argue that certain
members of the population are more or less likely to be sampled than others.
Note: there will usually be many valid answers for parts (a) and (b), but your
answer to part (b) will depend on how you define the population in part (a).
}


\question You need a survey on household spending patterns.  You take a random
sample from the customer list of the local brokerage firm.

\begin{solution}
\textit{Population: the spending patterns of all households.}
People who have brokerage accounts are certainly more wealthy
than average, so richer people are more likely to be sampled.
\end{solution}

\vspace{\stretch{1}}



\question You want to learn about New York City residents' sentiments (positive
or negative) towards their mayor, Bill~de~Blasio.  You search for
``de~Blasio'' on Twitter and read the first 100 relevant search results.

\begin{solution}
\textit{Population: the sentiments towards de~Blasio for all New York City
residents.}   People who use twitter are more likely to be sampled than people
who don't.  People who have tweeted about de~Blasio are more likely to be
sampled.  Furthermore, even among all people who use twitter, and have tweeted about
de~Blasio, those with recent tweets are more likely to be sampled.
\end{solution}

\vspace{\stretch{1}}


\question You need to know the opinions of Stern~undergraduate students with regard to
some curriculum matters.  You ask some of the people in your class.

\begin{solution}
\textit{Population: the opinions of all Stern~undergraduate students.}
Juniors and Seniors are more likely to be included.
\end{solution}

\vspace{\stretch{1}}


\question You want to learn about the quality of the food at a local
restaurant.  You read the reviews for the restaurant on Yelp.com.

\begin{solution}
\textit{Population: opinions of all people who have eaten at the restaurant.}
The people who write reviews on Yelp.com are more likely to be sampled than
people who do not use Yelp, or do not have Yelp accounts.  (Yelp reviewers
tend to have extreme opinions towards food, or tend to be overly critical.)
\end{solution}

\vspace{\stretch{1}}

\question You want to estimate the rate of growth of stocks over the last fifty
years.  You take a random sample of the stocks listed today on either the New
York Stock Exchange or the Nasdaq.  Some of these stocks did not exist fifty
years ago; you set these aside.  For the other stocks, you identify their
prices fifty years ago, and you use this to compute the growth rate.

\begin{solution}
\textit{Population: the rates of growth of all stocks over the last fifty
years.}
Companies that were listed fifty years ago but did not survive are not
available to appear in your sample. This is an example of survival bias. You
will seriously overestimate the growth rate!
\end{solution}

\vspace{\stretch{1}}

\newpage

\ifprintanswers\else\newpage\fi
  
 \question You want to know information about consumer preferences on a number of
 household products, including soap, laundry detergents, dishwashing
 detergents, furniture polish, and cleanser.   You devise a questionnaire item
 with 50 questions;  this takes ten to fifteen minutes to administer over the
 phone.  You randomly select phone numbers, and you get the responses of those
 who are home and willing to help you.
 
 \begin{solution}
 \textit{Population: the opinions of all consumers.}
 You have here a bias in favor of people who are at home and answer their
 phone. Such people may have non-typical opinions about consumer products.
 (Even if you were dealing with a non-home related topic, such as recent
 movies, you would still have a biased sample.) Finally, you are only getting
 the opinions of people who are willing to waste ten to fifteen minutes of
 their time talking to you! Why do we care about the opinions of such people?
 \end{solution}
 
 \vspace{\stretch{1}}
 
 
 
 \question You want to know whether a certain teaching method improves the reading
 abilities of fourth-grade students.  You examine all the articles on this
 subject published in five major education journals in the last ten years.
 
 \begin{solution}
 \textit{Population: the reading ability improvements of all fourth-grade
 students exposed to the teaching method.}
 Journals have a prejudice toward publishing articles which show strong
 statistical relationships. Submitted articles which show that the method fails
 are not likely to be published. This is called \emph{publication bias}.
 \end{solution}
 
 \vspace{\stretch{1}}
 
 %\newpage
 
 
 \question You want to learn about lifestyle habits which lead to kidney cancer.   You
 take a random sample of patients from the list of an oncology practice, and
 you interview these people with regard to issues like diet, cigarette smoking,
 chemical exposure, and so on.
 
 \begin{solution}
 \textit{Population: The lifestyle habits of all people who get kidney cancer.}
 This is clear survivor bias.  You are more likely to sample cancer survivors
 and people who live with kidney cancer for longer periods of time.
 \end{solution}
 
 \vspace{\stretch{1}}
 
 
 \question You want to get information about some mutual funds, so you research every
 fund which was advertised in the last four issues of a financial newsletter.
 
 \begin{solution}
 \textit{Population: information about all mutual funds.}
 This is a variant on publication bias. You will only get to see ads from
 funds that have done very well in the recent past.
 \end{solution}
 
 \vspace{\stretch{1}}
 
 
 \question You want to learn opinions among parents in your school district regarding
 adult literacy education.  You send out a letter inviting all parents inviting
 them to attend an information session.
 
 \begin{solution}
 \textit{Population: opinions of all parents in your school district.}
 This is selection bias. Obviously the illiterate will not be reading this
 letter about the meeting!
 \end{solution}
 
 \vspace{\stretch{1}}







\end{questions}

\end{document}

