\documentclass[answers,11pt]{exam}

\makeatletter
\newcommand{\course}[1]{\def\@course{#1}}
\makeatother

\makeatletter
\pagestyle{headandfoot}
\firstpageheader{}{\large \textbf{\@title \ifprintanswers ~-- Solutions \else \fi} \\ \@course}{}
\makeatother


\usepackage{amsmath}
\usepackage{amssymb}
\usepackage{booktabs}
\usepackage{graphicx}
\usepackage{subfig}

\DeclareGraphicsExtensions{.png,.pdf}


\DeclareMathOperator*{\Prob}{P}
\renewcommand{\Pr}{\Prob}
\DeclareMathOperator*{\E}{E}
\DeclareMathOperator*{\var}{var}
\DeclareMathOperator*{\sd}{sd}


\title{The Normal Model}
\course{STAT-UB.0001 -- Statistics for Business Control}


\begin{document}
\begin{questions}

  
\fullwidth{\section*{Standard Normal Random Variables}}


\question  Suppose $Z$ is a standard normal random variable.  What is $P(Z \leq
1.2)$?

\begin{solution}
\[
  P(Z \leq 1.2) = \Phi(1.2) = .8849.
\]
We have computed $\Phi(z)$ by using a normal table.
\end{solution}

\vspace{\stretch{1}}


\question  Suppose $Z$ is a standard normal random variable.  What is $P(Z \leq
-2.4)$?

\begin{solution}
\[
  P(Z \leq -2.4) = \Phi(-2.4) = .008198.
\]
We have computed $\Phi(z)$ by using a normal table.
\end{solution}

\vspace{\stretch{1}}


\question Suppose $Z$ is a standard normal random variable.  What is $P(Z \leq
-0.4)$

\begin{solution}
\[
  P(Z \leq -0.4) = \Phi(-0.4) = .3446.
\]
\end{solution}

\vspace{\stretch{1}}


\question Suppose $Z$ is a standard normal random variable.  What is $P(-0.4 \leq
Z \leq 1.2)$?

\begin{solution}
\begin{align*}
  P(-0.4 \leq Z \leq 1.2) &= P(Z \leq 1.2) - P(Z \leq -0.4) \\
  	&= \Phi(1.2) - \Phi(-0.4) \\
	&= .8849 - .3446 \\
	&= .5403.
\end{align*}
\end{solution}

\vspace{\stretch{1}}


\question Suppose $Z$ is a standard normal random variable.  What is $P(Z > 2)$?

\begin{solution}
\begin{align*}
  P(Z > 2) &= 1 - P(Z \leq 2) \\
  	      &= 1 - \Phi(2) \\
	      &= 1 - .97725 \\
	      &= 0.02275
\end{align*}
\end{solution}

\vspace{\stretch{1}}


\newpage

\fullwidth{\section*{Normal Cumulative Distribution Function (CDF)}}


\question The dressed weights of Excelsior Chickens are approximately normally
distributed with mean 3.20 pounds and standard deviation 0.40 pound.  About
what proportion of the chickens have dressed weights greater than 3.60 pounds?

\begin{solution}
Let $X$ be the weight of a typical chicken in pounds; this is normally distributed with
mean $\mu = 3.20$ and standard deviation $\sigma = 0.40$.  The proportion of
chickens with dressed weights greater than 3.60 is equal to the probability
that $X$ is greater than 3.60 pounds.

Define
$Z = (X - \mu)/\sigma$, a standard normal random variable.  Then,
\begin{align*}
  \Pr(X > 3.60)
    &= \Pr\Big(\frac{X - \mu}{\sigma} > \frac{3.60 - \mu}{\sigma}\Big) \\
    &= \Pr\Big(Z > \frac{3.60 - 3.20}{0.40}\Big) \\
    &= \Pr(Z > 1) \\
    &= 1 - \Pr(Z \leq 1) \\
    &= 1 - \Phi(1) \\
    &= 1 - .8413 \\
    &= .1587.
\end{align*}
\end{solution}

\vspace{\stretch{1}}



\ifprintanswers\newpage\fi

\question Suppose that an automobile muffler is designed so that its lifetime (in
months) is approximately normally distributed with mean 26 months and standard
deviation 4 months.  The manufacturer has decided to use a marketing strategy
in which the muffler is covered by warranty for 18 months.  Approximately what
proportion of the mufflers will fail before the warranty expires?

\begin{solution}
Let $X$ be the lifetime of a typical muffler in months; this is normally distributed
with mean $\mu = 26$ and standard deviation $\sigma = 4$.  The muffler will
fail before the warranty expires if and only if $X < 18$.  Thus,
the proportion of the mufflers that fail before the warranty expires is
equal to the probability that $X$ is less than 18.  Define $Z = (X -
\mu)/\sigma$, a standard normal random variable.  Then,

\begin{align*}
  \Pr(X < 18) &= P\Big(\frac{X - \mu}{\sigma} < \frac{18-26}{4}\Big) \\
  	&= P(Z < -2) \\
	&= \Phi(-2) \\
	&= .02275.
\end{align*}
\end{solution}

\vspace{\stretch{1}}


\question Suppose that the daily demand for change (meaning coins) in a
particular store is approximately normally distributed with mean \$800.00 and
standard deviation \$60.00.  What is the probability that, on any particular
day, the demand for change will be below \$600?

\begin{solution}
Let $X$ be the demain for change on a particular day (in dollars); this is a
normal random variable with mean $\mu = 800$ and standard deviation $\sigma =
60$.  Now
\[
  \Pr(X < 600) = \Pr\left(\frac{X - \mu}{\sigma} < \frac{600 - 800}{60}\right)
  = \Phi(-3.3) = .0004834.
\]
\end{solution}

\vspace{\stretch{1}}



\newpage


\fullwidth{\section*{Inverse Normal CDF}}




\question Suppose that the daily demand for change (meaning coins) in a
particular store is approximately normally distributed with mean \$800.00 and
standard deviation \$60.00.  Find the amount $M$ of change to keep on hand if
one wishes, with certainty 99\%, to have enough change.  That is, find $M$ so
that $\Pr(X \leq M) = 0.99$.

\begin{solution}
We have
\[
  .99 = P(X \leq M) = P\left(\frac{X - \mu}{\sigma} \leq \frac{M - 800}{60}\right) =
  \Phi\left(\frac{M - 800}{60}\right).
\]
Thus,
\[
  \frac{M - 800}{60} = \Phi^{-1}(.99).
\]
Consulting the normal inverse CDF table, we see that $\Phi(2.3263) = .99$ and
We take $\Phi^{-1}(.99) = 2.3263$, so that
\[
  M = 800 + 60 \times 2.3263 = 939.578.
\]
\end{solution}

\vspace{\stretch{1}}


\question Suppose that $Z$ is a standard normal random variable.  %\\
Find the value $w$ so that $\Pr(|Z| \leq w) = 0.60$.

\begin{solution}
The problem is asking for $w$ such that $\Pr(-w \leq Z \leq w) = 0.60$.  We
can look up this value directly, in the fourth column, getting
$w = .8416$.

For an alternative solution, note that
\[
  \Pr(Z < -w) + \Pr(-w \leq Z \leq w) + P(Z > w) = 1.
\]
Also, $\Pr(Z < -w) = \Pr(Z > w)$ (draw a picture if this is not obvious to you).
In this case, we must have that $\Pr(Z < -w) = 0.20$.  Therefore, $\Pr(Z < w) =
0.80$.

Now, $\Phi(w) = 0.80$.  The normal CDF table tells us that $\Phi(0.8416) =
0.80$; we take $w = 0.8416$.
\end{solution}

\vspace{\stretch{1}}

%\question Suppose that the latent load charge threshold for a population of
%investors is approximately normally distributed with mean 3.2 percentage
%points and standard deviation 0.8 of a percentage point.  Find the lower 10\%
%point.  That is, find the load rate $A$ so that, with probability 90\%, an
%investor will happily tolerate rate $A$.
%
%{\addtolength{\leftskip}{2.5em}
%
%Note: Each person has an upper limit for the load charge; if Dave's limit is 3.8\%,
%then he would not object to paying 3.1\%.  We can observe how any individual
%will behave at any specified rate, but we can't observe the individual's upper
%limit.  For this reason, we call the limit \emph{latent}.
%
%}


\question A machine that dispenses corn flakes into packages provides amounts that are
approximately normally distributed with mean weight 20 ounces and standard
deviation 0.6 ounce.  Suppose that the weights and measures law under which
you must operate allows you to have only 5\% of your packages under the weight
stated on the package.  What weight should you print on the package?

\begin{solution}
Let $X$ be the weight in ounces of a typical package; this is approximately
normally distributed with mean $\mu = 20$ and standard deviation $\sigma =
0.6$.  We seek a printed weight, $w$, such that $P(X < w) = .05$.  Define
$Z = (X - \mu)/\sigma$, a standard normal random variable.  We have
the following relation:
\[
  .05
  = \Pr(X < w) 
  = \Pr\left(\frac{X - \mu}{\sigma} < \frac{w - 20}{0.6}\right)
  = \Pr\left(Z < \frac{w - 20}{0.6}\right)
  = \Phi\left(\frac{w - 20}{0.6}\right).
\]
Thus,
\[
  \frac{w - 20}{0.6} = \Phi^{-1}(.05).
\]
With a normal table, we compute $\Phi(-1.6449) = 0.05$.
Finally,
\[
  \frac{w - 20}{0.6} = -1.6449,
\]
so $w = 20 - 0.6 \times 1.6449 = 19.01306$.  We would probably round this to
$19$ and print ``19 ounces'' on the box.
\end{solution}

\vspace{\stretch{1}}

% \newpage
% 
% \fullwidth{\section*{More examples}}
% 
% \question An industrial process produces five-liter cans of paint thinner.  The
% history of this process indicates a mean fill of 5.02 liters, with a standard
% deviation of 0.21 liter.  The quality control experts watch this process and
% select a can for inspection every hour.  This process runs for 12 hours every
% day.  The exact contents of the selected can are then determined.  The process
% is said to be ``in control'' if the volume is within the range $5.02 \pm a(0.21)$.
% 
% \begin{parts}
% 
% \part Suppose that $a = 2.0$.  About how often \emph{by chance alone}, will a
% can be declared out of control?
% 
% \begin{solution}
% Let $X$ be the volume (in liters) of a typical can; this is approximately a normal random
% variable with mean $\mu = 5.02$ and standard deviation $\sigma = 0.21$.  The
% proportion of out-of-control cans is equal to the probability that $X$ is
% outside the range $5.02 \pm a(0.21)$.
% 
% Note that $X$ is in control when $5.02 - a (0.21) \leq X \leq 5.02 + a
% (0.21)$.  Thus,
% \begin{align*}
%   \Pr(\text{$X$ is out} &\text{ of control}) \\
%     &= 1 - \Pr(\text{$X$ is in conrol}) \\
%     &= 1 - \Pr\big(5.02 - a (0.21) \leq X \leq 5.02 + a (0.21)\big) \\
%     &= 1 - \Pr\left(\frac{\big(5.02 - a (0.21)\big) - \mu}{\sigma}
%                     \leq \frac{X - \mu}{\sigma}
%                     \leq \frac{\big(5.02 + a (0.21)\big) - \mu}{\sigma}\right) \\
%     &= 1 - \Pr(-a \leq Z \leq a) \\
% \end{align*}
% where $Z = (X - \mu)/ \sigma$, a standard normal random variable.
% We compute
% \[
%   \Pr(-a \leq Z \leq a) = \Phi(a) - \Phi(-a).
% \]
% Thus, when $a = 2$,
% \[
%   \Pr(\text{$X$ is out of control}) =
%     1 - (.9772 - .0228) = 0.0456.
% \]
% \end{solution}
% 
% \vspace{\stretch{1}}
%   
% 
% \part Repeat this for $a = 2.5$ and $a = 3.0$.
% 
% \begin{solution}
% We can use the same logic as above.  For $a = 2.5$,
% \[
%   P(\text{$X$ is out of control}) =
%     1 - (.9938 - .0062) = .0124.
% \]
% For $a = 3.0$,
% \[
%   P(\text{$X$ is out of control}) =
%     1 - (.9987 - .0013) = .0026.
% \]
% \end{solution}
% 
% \vspace{\stretch{1}}
% 
% \end{parts}



% \question Suppose that an automobile muffler is designed so that its lifetime (in
% months) is approximately normally distributed with mean 26 months and standard
% deviation 4 months.  
% \begin{parts}
% \part The manufacturer has decided to use a marketing strategy
% in which the muffler is covered by warranty for 18 months.  Approximately what
% proportion of the mufflers will fail before the warranty expires?
% 
% \begin{solution}
% Let $X$ be the lifetime of a typical muffler in months; this is normally distributed
% with mean $\mu = 26$ and standard deviation $\sigma = 4$.  The muffler will
% fail before the warranty expires if and only if $X < 18$.  Thus,
% the proportion of the mufflers that fail before the warranty expires is
% equal to the probability that $X$ is less than 18.  Define $Z = (X -
% \mu)/\sigma$, a standard normal random variable.  Then,
% 
% \begin{align*}
%   \Pr(X < 18) &= P\Big(\frac{X - \mu}{\sigma} < \frac{18-26}{4}\Big) \\
%   	&= P(Z < -2) \\
% 	&= \Phi(-2) \\
% 	&= .0228.
% \end{align*}
% \end{solution}
% 
% \vspace{\stretch{1}}
% 
% 
% \part Suppose that the manufacturer in the previous example would like to
% extend the warranty time to 24 months.  Approximately what proportion of the
% mufflers will fail before the extended warranty expires?
% 
% \begin{solution}
% This is similar to part (a), but now we need to compute $P(X < 24)$:
% 
% \begin{align*}
%   \Pr(X < 24) &= P\Big(\frac{X - \mu}{\sigma} < \frac{24-26}{4}\Big) \\
%   	&= P(Z < -0.5) \\
% 	&= \Phi(-0.5) \\
% 	&= .3085.
% \end{align*}
% \end{solution}
% 
% \vspace{\stretch{1}}
%
% \part Of all the mufflers
% that fail under the extended warranty, what proportion of them have failures
% in the interval $(18 \text{ months}, 24 \text{ months})$?
% 
% 
% \begin{solution}
% The requested quantity is equal to the probability that $X$ is in the interval
% $(18, 24)$, \emph{conditional that $X$ is less than $24$}.  Using the
% definition of conditional probability, we compute
% \begin{align*}
%   \Pr(18 < X < 24 \mid X < 24)
%     &= \frac{\Pr(18 < X < 24 \text{ and } X < 24)}{\Pr(X < 24)} \\
%     &= \frac{\Pr(18 < X < 24)}{\Pr(X < 24)} \\
%     &= \frac{\Pr(X < 24) - \Pr(X < 18)}{\Pr(X < 24)} \\
%     &= \frac{.3085 - .0228}{.3085} \\
%     &= .9261.
% \end{align*}
% \end{solution}
% 
% \vspace{\stretch{1}}
% 
% \end{parts}




\end{questions}

\end{document}
