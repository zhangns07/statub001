\documentclass[11pt]{exam}

\makeatletter
\newcommand{\course}[1]{\def\@course{#1}}
\makeatother

\makeatletter
\pagestyle{headandfoot}
\firstpageheader{}{\large \textbf{\@title \ifprintanswers ~-- Solutions \else \fi} \\ \@course}{}
\makeatother


\usepackage{amsmath}
\usepackage{amssymb}
\usepackage{booktabs}
\usepackage{graphicx}
\usepackage{subfig}

\DeclareGraphicsExtensions{.png,.pdf}


\DeclareMathOperator*{\Prob}{P}
\renewcommand{\Pr}{\Prob}
\DeclareMathOperator*{\E}{E}
\DeclareMathOperator*{\var}{var}
\DeclareMathOperator*{\sd}{sd}
\renewcommand{\binom}[2]{\ensuremath{{}_{#1}C_{#2}}\,}



\title{Models for Counts}
\course{STAT-UB.0001 -- Statistics for Business Control}

\begin{document}
\begin{questions}

\fullwidth{\section*{Binomial Random Variables}}


\question A certain coin has a 25\% of landing heads, and a 75\% chance of
landing tails.

\begin{parts}
\part If you flip the coin 4 times, what is the chance of getting
exactly 2 heads?

\begin{solution}
There are 6 outcomes whith exactly 2 heads:
\[
  HHTT, HTHT, HTTH, THHT, THTH, TTHH.
\]
By independence, each of these outcomes has probability $(.25)^2 (.75)^2$.
Thus,
\[
  \Pr(\text{exactly 2 heads out of 4 flips}) = 6 (.25)^2 (.75)^2.
\]
\end{solution}

\vspace{\stretch{1}}


\part If you flip the coin 10 times, what is the chance of getting
exactly 2 heads?

\begin{solution}
Rather than list all outcomes, we will use a counting rule.  There are
$\binom{10}{2}$ ways of choosing the positions for the two heads; each of
these outcomes has probability $(.25)^2 (.75)^8$.  Thus,
\[
  \Pr(\text{exactly 2 heads out of 10 flips}) = \binom{10}{2} (.25)^2 (.75)^8.
\]
\end{solution}

\vspace{\stretch{1}}

\end{parts}

\question Suppose that you are rolling a die eight times.  Find the probability
that the face with two spots comes up exactly twice.

\begin{solution}
Let $X$ be the number of times that we get the face with two spots.
This is a binomial random variable with $n = 8$ and $p = \frac{1}{6}$.  We
compute
\begin{align*}
  P(X = 2)
    &= \binom{n}{2} p^2 (1 - p)^{n - 2} \\
    &= \binom{8}{2} \left(\frac{1}{6}\right)^2 \left(\frac{5}{6}\right)^{6} \\
    &\approx 0.26.
\end{align*}
\end{solution}

\vspace{\stretch{1}}


\newpage


\question The probability is 0.04 that a person reached on a ``cold call'' by a
telemarketer will make a purchase.  If the telemarketer calls 40 people, what
is the probability that at least one sale with result?

\begin{solution}
Let $X$ be the number of sales.  This is a binomial random variable with $n =
40$ and $p = 0.04$.  Thus,
\begin{align*}
  P(X \geq 1) &= 1 - P(X < 1) \\
    &= 1 - P(X = 0) \\
    &= 1 - \binom{n}{0} p^0 (1-p)^{n-0} \\
    &= 1 - (0.96)^{40} \\
    &\approx .805
\end{align*}
\end{solution}

\vspace{\stretch{1}}




\question A new restaurant opening in Greenwich village has a 30\% chance of
survival during their first year.  If 16 restaurants open this year, find the
probability that 

\begin{parts}
  
\part exactly 3 restaurants survive.

\begin{solution}
Let $X$ be the number that survive.  This is a binomial random variable with $n =
16$ and $p = 0.3$.  Therefore,
\begin{align*}
  P(X = 3)
  &= \binom{16}{3} (0.3)^3(1-0.3)^{(16 - 3)} \\
    &= .146
\end{align*}
\end{solution}

\vspace{\stretch{1}}

\part fewer than 3 restaurants survive.

\begin{solution}
\begin{align*}
  P(X < 3)
    &= P(X = 0) + P(X = 1) + P(X = 2) \\
    &= \binom{16}{0} (0.3)^0(0.7)^{16}
     + \binom{16}{1} (0.3)^1(0.7)^{15}
     + \binom{16}{2} (0.3)^2(0.7)^{14} \\
    &= .099
\end{align*}
\end{solution}

\vspace{\stretch{1}}

\part more than 3 restaurants survive.

\begin{solution}
\begin{align*}
  P(X > 3)
    &= 1 - P(X \leq 3) \\
    &= 1 - (.099 + .146) \\
    &= .754
\end{align*}
\end{solution}

\vspace{\stretch{1}}

\end{parts}

\newpage

\question \label{ques:binom-game} The probability of winning at a certain game
is 0.10.  If you play the game 10 times, what is the probability that you win
at most once?


\begin{solution}
Let $X$ be the number of times that we win.  This is a binomial random
variable with $n = 10$ and $p = 0.10$.  We compute
\begin{align*}
  P(X \leq 1)
    &= P(X = 0) + P(X = 1) \\
    &= \binom{n}{0} p^0 (1-p)^{n-0} + \binom{n}{1} p^1 (1 - p)^{n-1} \\
    &= \binom{10}{0} (0.10)^0 (0.90)^{10} + \binom{10}{1} (0.10)^1 (0.90)^9 \\
    &= (0.90)^{10} + 10 \, (0.10) (0.90)^9 \\
    &\approx 0.736.
\end{align*}

\end{solution}

\vspace{\stretch{1}}

\question The probability is 0.2 that an audit of a retail business will turn up
irregularities in the collection of state sales tax.  If 20 retail businesses
are audited, find the probability that

\begin{parts}

%\part exactly 3 will have irregularities in the collection of state sales tax.


%\part at least 3 will have irregularities in the collection of state sales
%tax.

%\begin{solution}
%We have
%\begin{align*}
%  P(X \geq 3)
%    &= 1 - P(X < 3) \\
%    &= 1 - \{ P(X = 0) + P(X = 1) + P(X = 2) \} \\
%    &= 1 - \left\{
%      \binom{16}{0} (0.3)^0 (0.7)^{16}
%      + \binom{16}{1} (0.3)^1 (0.7)^{15}
%      + \binom{16}{2} (0.3)^2 (0.7)^{14}
%      \right\}.
%\end{align*}
%\end{solution}

\part fewer than 2 will have irregularities in the collection of state sales
tax.

\begin{solution}
Let $X$ be the number audited.  This is a binomial random variable with $n =
20$ and $p = 0.2$.  Therefore,
\begin{align*}
  P(X < 2)
    &= \binom{20}{0} (0.2)^0 (0.8)^{20}
    + \binom{20}{1} (0.2)^1 (0.8)^{19}
    &\approx .069
\end{align*}
\end{solution}

\vspace{\stretch{1}}

%\item at most 5 will have irregularities in the collection of state sales tax.

\part more than 2 will have irregularities in the collection of state sales
tax.

\begin{solution}
\begin{align*}
  P(X > 2)
    &= 1 - P(X \leq 2) \\
    &= 1 - \bigg[ \binom{20}{0} (0.2)^0 (0.8)^{20}
    + \binom{20}{1} (0.2)^1 (0.8)^{19}
    + \binom{20}{2} (0.2)^2 (0.8)^{18} \bigg] \\
    &\approx .794.
\end{align*}
\end{solution}

\vspace{\stretch{1}}

%\part no more than 5 will have irregularites in the collection of state sales
%tax.

%\part no fewer than 5 will have irregularities in the collection of state
%sales tax.

\end{parts}


\newpage
\fullwidth{\section*{Poisson Random Variables}}

\question \label{ques:station-calls} The number of calls arriving at the
Swampside Police Station follows a Poisson distribution with rate 4.6/hour.

\begin{parts}

\part What is the probability that exactly six calls will come between 8:00
p.m.~and 9:00 p.m.?

\begin{solution}
Let $X$ be the number of calls that arrive between 8:00 p.m.~and 9:00 p.m.
This is a Poisson random variable with mean
\[
  \lambda = E(X) = (4.6 \text{ calls/hour}) (1 \text{ hour}) = 4.6\text{ calls}.
\]
Thus,
\[
  P(X = 6) = \frac{\lambda^6}{6!} e^{-\lambda} = \frac{(4.6)^6}{6!} e^{-4.6}.
\]
\end{solution}

\vspace{\stretch{1}}


\part Find the probability that exactly 7 calls will come between 9:00
p.m.~and 10:30 p.m.

\begin{solution}
Let $X$ be the number of calls that arrive between 9:00 p.m.~and 10:30 p.m.
This is a Poisson random variable with mean
\[
  \lambda = E(X) = (4.6 \text{ calls/hour}) (1.5 \text{ hours}) = 6.9\text{ calls}.
\]
Thus,
\[
  P(X = 7) = \frac{\lambda^7}{7!} e^{-\lambda} = \frac{(6.9)^7}{7!} e^{-6.9}.
\]
\end{solution}

\vspace{\stretch{1}}

\end{parts}


\question \label{ques:pois-car-accidents} Car accidents occur at a particular
intersection in the city at a rate of about 2/year.


\begin{parts}
  
\part Estimate the probability of no accidents occurring in a 6-month period.

\begin{solution}
Let $X$ be the number of car accidents.  This is  Poisson random variable with
mean
\[
  \lambda = E(X) = (2 \text{ accidents/year}) (0.5 \text{ years}) = 1\text{ accident}.
\]
Thus,
\[
  P(X = 0) = \frac{\lambda^0}{0!} e^{-\lambda} = e^{-1} \approx .368.
\]
\end{solution}

\vspace{\stretch{1}}

\part Estimate the probability of two or more accidents occurring in a year.

\begin{solution}
Let $X$ be the number of car accidents.  This is  Poisson random variable with
mean
\[
  \lambda = E(X) = (2 \text{ accidents/year}) (1.0 \text{ years}) = 2\text{ accident}.
\]
Thus,
\begin{align*}
  P(X \geq 2) 
    &= 1 - P(X < 2) \\
    &= 1 - \bigg[ \frac{\lambda^0}{0!} e^{-\lambda} + \frac{\lambda^1}{1!} e^{-\lambda} \bigg] \\
    &\approx .594.
\end{align*}
\end{solution}

\vspace{\stretch{1}}

\end{parts}


\newpage


\fullwidth{\section*{Empirical Rule with Binomial and Poisson Random Variables}}

\question{If you flip a fair coin 100 times, would it be unusual to get 42
heads and 58 tails?}

\begin{solution}
Let $X$ be the number of heads.  Then, $X$ is binomial with $n = 100$ and $p =
0.5$.  Thus, its expecation and standard deviation are
\[
  \mu = n p  = (100) (0.5) = 50,
\]
and
\[
  \sigma = \sqrt{n p (1 - p)} = \sqrt{(100) (0.5) (1 - 0.5)} = 5.
\]
Since $n p \geq 15$ and $n (1 - p) \geq 15$, we can use the empirical rule to
approximate the distribution of $X$.  Thus, approximately 95\% of the time,
$X$ will be in the range $\mu \pm 2 \sigma$, or $(40, 60)$.  So, it would not
be unusual get observe $X = 42$.

\end{solution}

\vspace{\stretch{1}}

\question If $X$ is a Poisson random variable with $\lambda = 225$, would it be
unusual to get a value of $X$ which is less than 190?

\begin{solution}
Set
\begin{gather*}
  \mu = E(X) = \lambda = 255, \\
  \sigma = \mathrm{sd}(X) = \sqrt{\lambda} = 15.
\end{gather*}
Define $z$ to be the number of standard deviations above the mean that 190 is,
i.e.
\[
  190 = \mu + \sigma z.
\]
Then,
\[
  z = \frac{190 - \mu}{\sigma} = \frac{-35}{15} \approx -2.33.
\]
A value of $X$ which is below 190 is more than 2.33 standard deviations
below the mean of $X$.  The empirical rule tells us that observations more
than 2 standard deviations away from the mean are unusual (they occur less
than 95\% of the time).  Therefore, values
of $X$ below 190 are unusual.
\end{solution}

\vspace{\stretch{1}}


\question The probability is 0.10 that a person reached on a ``cold call'' by
a telemarketer will make a purchase.  If the telemarketer calls 200 people,
would it be unusual for them to get 30 purchases?

\begin{solution}
Let $X$ be the number of purchases.  This is a Binomial random variable with
size $n = 200$ and success probability $p = 0.10$.  Thus, the expectation and
standard deviation of $X$ are
\begin{align*}
  \mu &= n p = (200)(.10) = 20 \\
  \sigma &= \sqrt{n p (1 - p)} = \sqrt{(200)(.10)(.90)} = \sqrt{18} \approx 4.2
\end{align*}
Since $n p \geq 15$ and $n p (1 - p) \geq 15$, the distribution of $X$ can be
approximated by the empirical rule.  Using the empirical rule
approximation, 95\% of the time, $X$ will be in
the range $\mu \pm 2 \sigma$, or (11.6, 28.4), and 99.7\% of the time, $X$
will be in the range $\mu \pm 3 \sigma$, or (7.4, 32.6).  We would see $X \geq 30$ less
than 5\% of the time.  It would be unusual to see X = 30, but not highly
unusual.
\end{solution}

\vspace{\stretch{1}}



\end{questions}

\end{document}
