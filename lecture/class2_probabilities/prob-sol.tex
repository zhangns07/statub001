\documentclass[answers,11pt]{exam}

\makeatletter
\newcommand{\course}[1]{\def\@course{#1}}
\makeatother

\makeatletter
\pagestyle{headandfoot}
\firstpageheader{}{\large \textbf{\@title \ifprintanswers ~-- Solutions \else \fi} \\ \@course}{}
\makeatother



\usepackage{amsmath}
\usepackage{amssymb}
\usepackage{booktabs}
\usepackage{graphicx}
\usepackage{subfig}

\DeclareGraphicsExtensions{.png,.pdf}


\DeclareMathOperator*{\Prob}{P}
\renewcommand{\Pr}{\Prob}
\DeclareMathOperator*{\E}{E}
\DeclareMathOperator*{\var}{var}
\DeclareMathOperator*{\sd}{sd}


\title{Probability}
\course{STAT-UB.0001 -- Statistics for Business Control}

\begin{document}

\begin{questions}


\fullwidth{\section*{Sample Points and Sample Spaces}}

\question In the following two experiments, what are the sample points and the
sample space?

\begin{parts}
\part You flip a coin.

\begin{solution}
The sample points are $H$, ``the outcome is heads,'' and $T$, ``the outcome is
tails.'' The sample space is the set of all sample points: $\Omega = \{ H, T \}$.
\end{solution}
\vspace{\stretch{0.5}}


\part You roll a 6-sided die.

\begin{solution}
The sample points are the possible outcomes of the die: 1, 2, 3, 4, 5, 6.  The
sample space is the set of all sample points: $\Omega = \{ 1, 2, 3, 4, 5, 6 \}$.
\end{solution}

\vspace{\stretch{0.5}}

\end{parts}


\question Suppose that a customer visits a restaurant and leaves a review on
Yelp with 1--5 stars.  What are the sample points and the sample space for the
customer's star rating?

\begin{solution}
The sample points are the possible star ratings: 1, 2, 3, 4, 5.  The sample
space is the set of all sample points: $\Omega = \{ 1, 2, 3, 4, 5 \}$.
\end{solution}

\vspace{\stretch{1}}


\question Suppose that two customers visit a restaurant, and that they both
leave Yelp reviews with 1--5 stars each.
What are the sample points and the sample space for the pair of star ratings?

\begin{solution}
Each sample point can be represented by an ordered pair $(i,j)$, where $i$ is
the first customer's star rating, and $j$ is second customer's star rating.  The
sample points are the elements of the following table.
\begin{center}
\begin{tabular}{c|c|c|c|c|}
\multicolumn{1}{c}{} &
\multicolumn{1}{c}{1} &
\multicolumn{1}{c}{2} &
\multicolumn{1}{c}{$\cdots$} &
\multicolumn{1}{c}{5} \\
\cline{2-5}
1 & (1,1) & (1,2) & $\cdots$ & (1,5) \\
\cline{2-5}
2 & (2,1) & (2,2) & $\cdots$ & (2,5) \\
\cline{2-5}
$\vdots$ & $\vdots$ & $\vdots$ & $\ddots$ & $\vdots$ \\
\cline{2-5}
5 & (5,1) & (5,2) & $\cdots$ & (5,5) \\
\cline{2-5}
\end{tabular}
\end{center}
The sample space is the set of all 25 sample points: $\Omega = \{ (1,1), (1,2),
\dotsc, (5,5) \}$.
\end{solution}

\vspace{\stretch{1}}


\newpage
\question Suppose you randomly pick a respondent from the class survey, then
record their undergraduate major and gender.  What are the sample points and the sample space?
Assume that major is either ``Business,'' ``Humanities/Social Science,'' or
``Science/Engineering.''

\begin{solution}
Each sample point is a gender-major pair.  The sample space is the set of all
possible gender-major pairs: $\Omega = \{ (\text{Bus.}, \text{Female}),$
$(\text{Bus.}, \text{Male}),$
$(\text{Hum./Soc.~Sci.}, \text{Female}),$
$(\text{Hum./Soc.~Sci.}, \text{Male}),$
$(\text{Sci./Eng.}, \text{Female}),$
$(\text{Sci./Eng.}, \text{Male}) \}$.
Note that the sample space is all \emph{possible} gender-major pairs, not all
\emph{observed} gender-major pairs.
\end{solution}

\vspace{\stretch{1}}


\fullwidth{\section*{Events}}


\question \label{ques:odd-yelp} Suppose that a customer leaves a Yelp
rating (1--5 stars) for a restaurant.  Describe the event ``the rating is odd
(not even).''

\begin{solution}
\[
  O = \{ 1, 3, 5 \}.
\]
\end{solution}



\question \label{ques:survey-male}

Suppose you randomly pick a respondent from the class survey, then record
their undergraduate major and gender.  Assume that undergraduate major is
listed as ``Business'', ``Hum./Soc.~Sci.'', or ``Sci./Eng.'', and that gender
is listed as ``Male'' or ''Female''.


\begin{parts}

\part List the sample points in the event ``the major is Business.''

\begin{solution}
\(
  \text{Business} = \{ (\text{Business}, \text{Female}), (\text{Business}, \text{Male}) \}.
\)
\end{solution}

\vspace{\stretch{0.5}}


\part List the sample points in the event ``the gender is Male.''

\begin{solution}
\(
  \text{Male} = \big\{
    \big(\text{Business}, \text{Male}\big),
    \big(\text{Hum./Soc.~Sci.}, \text{Male}\big),
    \big(\text{Sci./Eng.}, \text{Male}\big)
  \big\}.
\)
\end{solution}

\vspace{\stretch{0.5}}

\end{parts}

\newpage



\fullwidth{\section*{Probability}}


\question Suppose you randomly pick a respondent from the class survey and
record their undergraduate major and gender.

\begin{parts}

\part Use the following table of recorded survey
response frequencies to compute the probabilities of the sample points.

\vspace{\baselineskip}

\begin{center}
\bgroup
%\def\arraystretch{1.5}
\begin{tabular}{l@{\extracolsep{2em}}rrr}
\toprule
& \multicolumn{2}{c}{Gender} & \\
\cmidrule(r){2-3}
Undergrad Major & Female & Male & Total \\
\midrule
Business       &  9 &  6 & 15 \\
Hum./Soc.~Sci. & 10 & 12 & 22 \\
Sci./Eng.      &  2 &  8 & 10 \\
\addlinespace
\quad Total    & 21 & 26 & 47 \\
\bottomrule
\end{tabular}
\egroup
\end{center}

\vspace{\baselineskip}

\begin{solution}
To compute the probabilities for the 6 sample points corresponding to the cells
of the table, we take the recorded frequency and divide by the total number of
survey respondents.  We have
\begin{align*}
  \Pr\big(\text{Bus.},       \text{Female}\big) &= \frac{ 9}{47} \approx .19, \\
  \Pr\big(\text{Bus.},       \text{Male}  \big) &= \frac{ 6}{47} \approx .13, \\
  \Pr\big(\text{Hum./Soc.~Sci.}, \text{Female}\big) &= \frac{10}{47} \approx .21, \\
  \Pr\big(\text{Hum./Soc.~Sci.}, \text{Male}  \big) &= \frac{12}{47} \approx .26, \\
  \Pr\big(\text{Sci./Eng.},         \text{Female}\big) &= \frac{ 2}{47} \approx .04, \\
  \Pr\big(\text{Sci./Eng.},         \text{Male}  \big) &= \frac{ 8}{47} \approx .17.
\end{align*}
\end{solution}

\vspace{\stretch{1}}


\part Find the probability that the undergraduate major is Business.

\begin{solution}
\begin{align*}
\Pr(\text{Business}) &= \tfrac{9}{47} + \tfrac{6}{47} \\
       &= \tfrac{15}{47} \\
       &\approx 32\%.
\end{align*}
\end{solution}

\vspace{\stretch{1}}

\part Find the probability that the gender is Male.

\begin{solution}
\begin{align*}
\Pr(\text{Male}) &= \tfrac{6}{47} + \tfrac{12}{47} + \tfrac{8}{47} \\
       &= \tfrac{26}{47} \\
       &\approx 55\%.
\end{align*}
\end{solution}

\vspace{\stretch{1}}


\part Find the probability the undergraduate major is
Humanities/Social~Science.

\begin{solution}
\begin{align*}
\Pr(\text{Hum./Soc.~Sci.}) &= \tfrac{10}{47} + \tfrac{12}{47} \\
       &= \tfrac{22}{47} \\
       &\approx 47\%.
\end{align*}

\end{solution}
\vspace{\stretch{1}}

\end{parts}



\question Suppose that a customer's Yelp rating is random, and that the
probabilities for the possible star ratings are
\(
  p_1 = 10\%,
\)
\(
  p_2 = 30\%,
\)
\(
  p_3 = 25\%,
\)
\(
  p_4 = 20\%,
\)
\(
  p_5 = 15\%.
\)
Find the probability that the rating is odd.

\begin{solution}
We add up the probabilities of the sample points in the event:
\begin{align*}
  \Pr(\{1, 3, 5\})
    &= p_1 + p_3 + p_5 \\
    &= 10\% + 25\% + 15\% \\
    &= 50\%.
\end{align*}
\end{solution}

\vspace{\stretch{1}}


\newpage



\fullwidth{\section*{Compound Events and the Additive Rule}}

\question Suppose you pick a random survey respondent and record their
undergraduate major and gender.  

\begin{parts}

\part List the sample points in the event
``the major is Business or the gender is Male.''

\begin{solution}
Denote the event by $A$.  Then,
\begin{multline*}
  A = \{
  (\text{Business}, \text{Female}), (\text{Business}, \text{Male}),
            (\text{Hum./Soc.~Sci.}, \text{Male}),
            (\text{Sci./Eng.}, \text{Male})
      \}.
\end{multline*}
\end{solution}

\vspace{\stretch{1}}

\part Compute the probability of the event in part (a) by adding the
probabilities of all of the sample points in the event.

\begin{solution}
\begin{align*}
  \Pr(A)
  &=
  \tfrac{9}{47}
  + \tfrac{6}{47}
  + \tfrac{12}{47}
  + \tfrac{8}{47} \\
  &= \tfrac{35}{47} \\
  &\approx 74\%.
\end{align*}
\end{solution}


\vspace{\stretch{1}}

\part Express the event ``the major is Business or the gender is Male'' as a
union of two other events.

\begin{solution}
\begin{align*}
  A &= \{ \text{major is Business or gender is Male} \} \\
    &= \text{Business} \cup \text{Male}.
\end{align*}
\end{solution}

\vspace{\stretch{1}}


\part Compute the probability of the event using the additive rule.

\begin{solution}
\begin{align*}
  \Pr(A)
  &= \Pr(\text{Business} \cup \text{Male}) \\
  &= \Pr(\text{Business}) + \Pr(\text{Male})
      - \Pr(\text{Business} \cap \text{Male}) \\
  &= \tfrac{15}{47} + \tfrac{26}{47} - \tfrac{6}{47} \\
  &= \tfrac{35}{47} \\
  &\approx 74\%.
\end{align*}
\end{solution}
\vspace{\stretch{1}}


\end{parts}


\newpage
\question Suppose that two customers give ratings (1--5 stars) to the same restaurant
on Yelp.  
\label{ques:yelp-at-least-1}

\begin{parts}
  
\part Express the event ``at least one customer gives a 1 star rating'' as a
union of two other events.

\begin{solution}
\begin{multline*}
  A = \{ \text{ the first customer gives a 1 star rating } \} \\
      \cup \{ \text{ the second customer gives a 1 star rating } \}.
\end{multline*}
\end{solution}

\vspace{\stretch{1}}

\part Suppose that both customers randomly assign their ratings, giving equal
probabilities to all possible star ratings.  In this case, all 25 sample
points have equal probability.  Compute the probability of the event in part (a).

\begin{solution}
Using the additive rule,
\begin{align*}
\begin{split}
  \Pr(A) &= \Pr(\text{1 from first customer})
          + \Pr(\text{1 from second customer}) \\
          &\quad- \Pr(\text{1 from the first customer \emph{and} 1 from second customer})
\end{split} \\
         &= \tfrac{1}{5} + \tfrac{1}{5} - \tfrac{1}{25} \\
         &= \tfrac{9}{25} \\
         &= 36\%.
\end{align*}
\end{solution}
\vspace{\stretch{1}}

\end{parts}

\vspace{\stretch{0.5}}

\question Suppose that two customers give ratings to the same restaurant on
Yelp.  

\begin{parts}
  
\part Express the event ``the average of their ratings is 3.5 or 4'' as a
union of two other events. \\
\textit{Hint: this is the same event as ``the sum of their ratings is 7 or
8.''}


\begin{solution}
Define two events:
\begin{align*}
  S_7 &= \{ \text{ the sum of their ratings is 7 } \} \\
      &= \{ (2,5), (3,4), (4,3), (5,2) \}, \\
  S_8 &= \{ \text{ the sum of their ratings is 8 } \} \\
      &= \{ (3,5), (4,4), (5,3)  \}.
\end{align*}
Then, the event we care about is $A = S_7 \cup S_8$.
\end{solution}

\vspace{\stretch{1}}


\part As in problem~\ref{ques:yelp-at-least-1}(b), 
suppose that both customers randomly assign their ratings with equal
probability for all possible star ratings, so that all 25 sample points
have equal probability.  Compute the probability of the event in part (a).

\begin{solution}
Using the additive rule,
\begin{align*}
  \Pr(A)
    &= \Pr(S_7 \cup S_8) \\
    &= \Pr(S_7) + \Pr(S_8) - \Pr(S_7 \cap S_8).
\end{align*}
We note that the sum can't be 7 and 8 simultaneously, so $S_7$ and $S_8$ are
mutually exclusive events, i.e.~$S_7 \cap S_8 = \emptyset$.  Thus,
\begin{align*}
  \Pr(A)
    &= \Pr(S_7) + \Pr(S_8) \\
    &= \tfrac{4}{25} + \tfrac{3}{25} \\
    &= \tfrac{7}{25} \\
    &= 28\%.
\end{align*}
\end{solution}
\vspace{\stretch{1}}

\end{parts}

\vspace{\stretch{1}}


\newpage

\fullwidth{\section*{Complementary Events and the Complement Rule}}


\question Suppose that 60\% of NYU undergraduates own iPhones.  If you pick a
random NYU undegraduate, what is the probability that he or she will \emph{not} own
an iPhone?

\begin{solution}
The sample space is the set of all students.  Let
\[
  A = \{ \text{ the randomly picked student owns an iPhone } \}.
\]
Then,
\[
  A^c = \{ \text{ the randomly picked student does not own an iPhone } \},
\]
so by the complement rule,
\begin{align*}
  \Pr(A^c) &= 1 - \Pr(A) \\
           &= 1 - .60 \\
           &= .40.
\end{align*}
\end{solution}

\vspace{\stretch{1}}


\question Suppose you flip five coins.  What is the probability of getting at
least one head?
\\
\textit{Hint: what is the complement of this event?}

\begin{solution}
The sample space, $\Omega$, is the set of all possible outcomes for the five flips.
Since there are 5 independent flips, and each has 2 possible outcomes, we have
that $|\Omega| = 2^5 = 32$.

Let
\[
  A = \{ \text{ you get at least one head } \}.
\]
Then,
\begin{align*}
  A^c &= \{ \text{ you don't get any heads } \} \\
      &= \{ (T,T,T,T,T) \}.
\end{align*}
Thus, by the complement rule,
\begin{align*}
  \Pr(A) &= 1 - \Pr(A^c) \\
         &= 1 - \tfrac{1}{32} \\
         &= \tfrac{31}{32}.
\end{align*}
\end{solution}
\vspace{\stretch{1}}



\newpage
\fullwidth{\section*{Conditional Probability}}



\question \label{ques:industry-gender}

Here is a table of the tabulated frequencies for undergrad major and gender for the respondents to a class survey.
\begin{center}
\bgroup
%\def\arraystretch{1.5}
\begin{tabular}{l@{\extracolsep{2em}}rrr}
\toprule
& \multicolumn{2}{c}{Gender} & \\
\cmidrule(r){2-3}
Undergrad Major & Female & Male & Total \\
\midrule
Business       &  9 &  6 & 15 \\
Hum./Soc.~Sci. & 10 & 12 & 22 \\
Sci./Eng.      &  2 &  8 & 10 \\
\addlinespace
\quad Total    & 21 & 26 & 47 \\
\bottomrule
\end{tabular}
\egroup
\end{center}


\begin{parts}

\part Express the following statements as conditional probabilities:
\begin{itemize}
\item $\frac{9}{21} \approx 43\%$ of the females have undergrad major in Business. 

\item $\frac{9}{15} = 60\%$ of those having undergrad major in Business  are female.
\end{itemize}

\begin{solution}
\begin{align*}
  P(\text{Business} \mid \text{Female}) &= \frac{9}{21}, \\
  P(\text{Female} \mid \text{Business} ) &= \frac{9}{15}.
\end{align*}
\end{solution}

\vspace{\stretch{1}}

\part Compute
  $P(\text{Male} \mid \text{Sci./Eng.})$
and
  $P(\text{Sci./Eng.}  \mid \text{Male})$.
Explain the difference between these two quantities.

\begin{solution}
\begin{align*}
  P(\text{Male} \mid \text{Sci./Eng.}) &= \frac{8}{10} = 80\%, \\
  P(\text{Sci./Eng.} \mid \text{Male}) &= \frac{8}{26} \approx 31\%.
\end{align*}
The quantity
  $P(\text{Male} \mid \text{Sci./Eng.} )$
is the proportion of those having undergrad major in Sci./Eng.  that are male.
The quantity
$P(\text{Sci./Eng.} \mid \text{Male})$
is the proportion of males having undergrad major in Sci./Eng.\end{solution}

\vspace{\stretch{1}}

\end{parts}

\newpage


\question The following table lists the pick-up and drop-off locations of
approximately 170 million yellow cab taxi trips made in New York City in 2013.
Numbers are reported in thousands.


\bgroup
\footnotesize
\begin{center}
\begin{tabular}{l@{\extracolsep{2em}}rrrrrr}
\toprule
& \multicolumn{5}{c}{Drop-off} & \\
\cmidrule(r){2-6}
Pick-up    & Bronx &  Brooklyn & Manhattan & Queens & Staten Is. & Total \\
\midrule
Bronx      &     53 &         1 &        37 &      4 &          0 &      95 \\
Brooklyn   &      8 &     2,707 &     1,598 &    273 &          2 &   4,588 \\
Manhattan  &    638 &     5,458 &   143,656 &  5,906 &         22 & 155,680 \\
Queens     &    122 &     1,022 &     5,058 &  2,281 &          8 &   8,491 \\
Staten Is. &      0 &         0 &         0 &      0 &          3 &       3 \\
\addlinespace
\quad Total &   821 &     9,188 &   150,349 &  8,464 &         35 & 168,857 \\
\bottomrule
\end{tabular}
\end{center}
\egroup

\begin{parts}


\part Find
$P(\text{drop-off Brooklyn} \mid \text{pick-up Manhattan})$
and
$P(\text{pick-up Manhattan} \mid \text{drop-off Brooklyn})$.  Explain the
difference between these two quantities.

\begin{solution}
\begin{align*}
  P(\text{drop-off Brooklyn} \mid \text{pick-up Manhattan})
    &= \frac{5458}{155680}
    \approx 3.5\%, \\
  P(\text{pick-up Manhattan} \mid \text{drop-off Brooklyn})
    &= \frac{5458}{9188}
    \approx 59.4\%.
\end{align*}
3.5\% of the rides that pick up in Manhattan drop off in Brooklyn;
59.4\% of the rides that drop off in Brooklyn originate in Manhattan.

\end{solution}
\vspace{\stretch{1}}

\part Express the following statement as a conditional probability: ``29\% of
the trips with drop-off locations in Brooklyn originated in the same
borough.''

\begin{solution}
\[
  P(\text{pick-up Brooklyn} \mid \text{drop-off Brooklyn}) = \frac{2707}{9188} = 29\%.
\]
Note:
\[
  P(\text{drop-off Brooklyn} \mid \text{pick-up Brooklyn}) = \frac{2707}{4588} = 59\%.
\]
\end{solution}

\vspace{\stretch{1}}

\end{parts}

\newpage



% \question \label{I:iphone} Suppose the following table gives the numbers of NYU undergraduates
% who own iPhones, broken down by class:
% 
% \begin{center}
% \bgroup
% %\def\arraystretch{1.5}
% \begin{tabular}{l@{\extracolsep{2em}}rrr}
% \toprule
% & \multicolumn{2}{c}{Own iPhone} & \\
% \cmidrule(lr){2-3}
% Class & Yes & No & Total \\
% \midrule
% Freshman & 400 & 600 & 1000 \\
% Sophomore & 300 & 600 & 900 \\
% Junior & 700 & 500 & 1200 \\
% Senior & 500 & 500 & 1000 \\
% \addlinespace
% \quad Total & 1900 & 2200 & 4100 \\
% \bottomrule
% \end{tabular}
% \egroup
% \end{center}
% 
% \begin{parts}
% 
% \part If we pick a random NYU undergrauate, and she happens to be a Freshman,
% what is the probability that she owns an iPhone?
% 
% \begin{solution}
% \[
%   \frac{400}{1000}
% \]
% \end{solution}
% \vspace{\stretch{1}}
% 
% 
% \part Do the above problem using the equation $\Pr(B \mid A) = \frac{\Pr(A \cap B)}{\Pr(A)}$.
% 
% \begin{solution}
% Define the following two events
% \begin{align*}
%   I &= \text{``the student owns an iPhone''} \\
%   F &= \text{``the student is a Freshman''}
% \end{align*}
% Then,
% \begin{align*}
%   \Pr(I \mid F)
%     &= \frac{\Pr(I \cap F)}{\Pr(F)} \\
%     &= \frac{400/4100}{1000/4100} \\
%     &= \frac{400}{1000}. 
% \end{align*}
% \end{solution}
% \vspace{\stretch{1}}
% 
% 
% \ifprintanswers\newpage\fi
% 
% \part If we pick a random NYU undergraduate and he happens to have an iPhone,
% what is the probability that he is a Freshman?
% 
% \begin{solution}
% \begin{align*}
%   \Pr(F \mid I)
%     &= \frac{\Pr(F \cap I)}{\Pr(I)} \\
%     &= \frac{400/4100}{1900/4100} \\
%     &= \frac{400}{1900}.
% \end{align*}
% \end{solution}
% \vspace{\stretch{1}}
% 
% 
% \part If we pick a random NYU undergraduate and he does not have an iPhone,
% what is the probability that he is a Sophomore?
% 
% \begin{solution}
% Define the event
% \[
%   S = \text{``the student is a Sophomore''}.
% \]
% Then,
% \begin{align*}
%   \Pr(S \mid I^c)
%     &= \frac{\Pr(S \cap I^c)}{\Pr(I^c)} \\
%     &= \frac{600/4100}{2200/4100} \\
%     &= \frac{600}{2200}.
% \end{align*}
% \end{solution}
% \vspace{\stretch{1}}
% 
% 
% \part If we pick a random NYU undergraduate, and we happen to pick an
% upperclassman (Junior or Senior), what is probability that she does not own
% an iPhone?
% 
% \begin{solution}
% Define the following events
% \begin{align*}
% %  J &= \{ \text{ the student is a Junior } \} \\
% %  T &= \{ \text{ the student is a Senior } \} \\
%   U &= \text{``the student is an upperclassman''}.
% \end{align*}
% Then,
% \begin{align*}
%   \Pr(I^c \mid U)
%     &= \frac{\Pr(I^c \cap U)}{\Pr(U)}
% \end{align*}
% Now,
% \begin{align*}
%   \Pr(U) %&= \Pr(J \cup T) \\
%          %&= \Pr(J) + \Pr(T) \\
%          &= \frac{1200 + 1000}{4100} \\
%          &= \frac{2200}{4100}.
% \end{align*}
% %where we have used the fact that $J$ and $T$ are mutually exclusive to compute
% %$\Pr(J \cup T)$.
% Next,
% \begin{align*}
%   \Pr(I^c \cap U)
% %  &= \Pr\big(I^c \cap (J \cup T)\big) \\
% %  &= \Pr\big((I^c \cap J) \cup (I^c \cap T)\big) \\
% %  &= \Pr(I^c \cap J) + \Pr(I^c \cap T) \\
%   &= \frac{500 + 500}{4100} \\
%   &= \frac{1000}{4100}.
% \end{align*}
% Putting it all together,
% \begin{align*}
%   \Pr(I^c \mid U)
%     &= \frac{\Pr(I^c \cap U)}{\Pr(U)} \\
%     &= \frac{1000/4100}{2200/4100} \\
%     &= \frac{1000}{2200}.
% \end{align*}
% \end{solution}
% \vspace{\stretch{1}}
% 
% 
% \end{parts}
% 
% \ifprintanswers\else\newpage\fi


%\question Suppose we roll two dice and $A = \text{``The sum is 8,''}$
%$B = \text{``The first die is 3.''}$  What is the probability that the first
%die is 3 given that the sum is 8, i.e. what is $P(B \mid A)$?
%
%\begin{solution}
%The sample space, $\Omega$, is the set of all 36 pairs $(i,j)$, where $i$ is
%the first roll, and $j$ is the second roll:
%\[
%  \Omega = \{ (1,1), (1,2), \dotsc, (6,6) \}.
%\]
%The events of interest are
%\begin{align*}
%  A &= \{ (2,6), (3,5), (4,4), (5,3), (6,2) \}, \\
%  B &= \{ (3,1), (3,2), (3,3), (3,4), (3,5), (3,6) \}, \\
%  A \cap B &= \{ (3,5) \}.
%\end{align*}
%Thus,
%\begin{align*}
%  \Pr(B \mid A)
%    &= \frac{\Pr(B \cap A)}{\Pr(A)} \\
%    &= \frac{1/36}{5/36} \\
%    &= \frac{1}{5}.
%\end{align*}
%\end{solution}
%
%\vspace{\stretch{1}}
%
%
%\question Suppose we draw to cards out of a deck of 52.  Let $A = \text{``The
%first card is a space,''}$ $B = \text{``The second card is a spade.''}$  Find
%the probability that the second card is a spade given that the first card is
%a spade.
%
%\begin{solution}
%If the first card is a spade, then at the time we draw the second card, there
%are 51 cards in the deck, and 12 of them are spades.  Thus,
%\[
%  \Pr(B \mid A) = \frac{12}{51}.
%\]
%The other way to do this problem is by first computing
%\begin{align*}
%  \Pr(A) &= \frac{13}{52} \\
%  \Pr(B \cap A) &= \frac{\binom{13}{2}}{\binom{52}{2}} \\
%                &= \frac{13 \cdot 12 / 2!}{52 \cdot 51 / 2!} \\
%                &= \frac{13 \cdot 12}{52 \cdot 51}.
%\end{align*}
%Then,
%\begin{align*}
%  \Pr(B \mid A)
%    &= \frac{\Pr(B \cap A)}{\Pr(A)} \\
%    &= \frac{12}{51}.
%\end{align*}
%\end{solution}
%\vspace{\stretch{1}}


\fullwidth{\section*{The Multiplicative Rule}}


\question Out of the 58 students enrolled in the class, 24 are female (41\%) and
34 are male (59\%).  Suppose that we randomly select two different students.

\begin{parts}

\part What is the probability that both students are male?

\begin{solution}
Define the two events
\begin{align*}
  A &= \text{the first student picked is male} \\
  B &= \text{the second student picked is male}.
\end{align*}
Then, $P(A) = \tfrac{34}{58}$, and $P(B\mid A) = \tfrac{33}{57}$.  Thus, the
probability that both will be male is
\begin{align*}
  P(A \cap B) &= P(A) P(B \mid A) \\
              &= \frac{34}{58} \cdot \frac{33}{57} \\
              &= \frac{1122}{3306} \\
              &\approx 34\%.
\end{align*}
\end{solution}

\vspace{\stretch{1}}


\part What is the probability that both students are female?

\begin{solution}
Using the events $A$ and $B$ defined in the previous part,
$P(A^c) = \tfrac{24}{58}$ and $P(B^c \mid A^c) = \tfrac{23}{57}$.  Thus, the
probability that both will be female is
\begin{align*}
  P(A^c \cap B^c) &= P(A^c) P(B^c \mid A^c) \\
                  &= \frac{24}{58} \cdot \frac{23}{57} \\
                  &= \frac{552}{3306} \\
                  &\approx 17\%.
\end{align*}
\end{solution}

\vspace{\stretch{1}}


\part What is the probability that one of the students is male and one of the
students is female?

\begin{solution}
The event ``one student is male and the other is female'' is equivalent to
the compound event $(A \cap B^c) \cup (A^c \cap B)$; that is, either the
first is male and the second is female, or the first is female and the
second is male.  Since $A \cap B^c$ and $A^c \cap B$ are mutually exclusive,
it follows that
\[
  P(\text{one male and one female})
    = P(A \cap B^c) + P(A^c \cap B).
\]
Using the multiplicative rule,
\begin{align*}
  P(A \cap B^c) &= P(A) P(B^c \mid A) \\
                &= \frac{34}{58} \frac{24}{57} \\
                &= \frac{816}{3306} \\
  \\
  P(A^c \cap B) &= P(A^c) P(B \mid A^c) \\
                &= \frac{24}{58} \frac{34}{57} \\
                &= \frac{816}{3306}.
\end{align*}
Thus,
\begin{align*}
  P(\text{one male and one female})
    &= \frac{816}{3306} + \frac{816}{3306} \\
    &= \frac{1632}{3306} \\
    &\approx 49\%.
\end{align*}
\end{solution}

\vspace{\stretch{1}}

\end{parts}



\question Of the 48 students who filled out the survey, 33 indicated that they
drink at least one cup of coffee per day, while 15 indicated that they do not
drink coffee on a typical day.  Suppose that we randomly select two different survey respondents.

\begin{parts}

\part What is the probability that both students regularly drink coffee?

\begin{solution}
\[
  \frac{33}{48} \cdot \frac{32}{47} = \frac{1056}{2256} \approx 47\%.
\]
\end{solution}

\vspace{\stretch{1}}


\part What is the probability that neither student regularly drinks coffee?

\begin{solution}
\[
  \frac{15}{48} \cdot \frac{14}{47} = \frac{210}{2256} \approx 9\%.
\]
\end{solution}

\vspace{\stretch{1}}


\part What is the probability that exactly one student regularly drinks
coffee?

\begin{solution}
\[
  \frac{33}{48} \cdot \frac{15}{47} + \frac{15}{48} \cdot \frac{33}{47}
    = \frac{990}{2256}
    \approx 44\%.
\]
\end{solution}

\vspace{\stretch{1}}

\end{parts}

\newpage

\question A class has 20 students.  What is the probability that at least
two students have the same birthday?  Assume that each person in the class was
assigned a random birthday between January 1 and December 31.

\begin{solution}
Assume that everyone in the class is randomly assigned a birthday, which
corresponds to number between 1 and 365 representing the day of the year.
It turns out to be much easier to compute the probability using the complement
rule, as
\[
  \Pr(\text{at least 2 people have the same birthday})
  = 1 - \Pr(\text{all 20 birthdays are different}).
\]
The next trick is to write the event that all 50 birthdays are different in a
redundant way:
\begin{align*}
\{\text{all 50} &\text{ birthdays are different}\}
  = \{ \text{first 2 are different} \}
  \cap \{ \text{first 3 are different} \} \\
  &\quad\cap \{ \text{first 4 are different} \}
  \cap \{ \text{first 5 are different} \}
  \cap \dotsb \cap \{ \text{first 50 are different} \}.
\end{align*}

%First, we use the multiplicative rule to write
%\begin{align*}
%\Pr(\text{all 70} &\text{ birthdays are different}) \\
%  &= \Pr(\text{first 2 diff.})
%     \Pr(\{ \text{first 3 diff.} \}
%  \cap \dotsb \cap \{ \text{first 70 diff.} \}
%  \mid
%  \text{first 2 diff.} ).
%\end{align*}
%Next, we use the multiplicative rule on the second term:
%\begin{align*}
%  \Pr(\{ \text{first 3 diff.} \}
%  &\cap \dotsb \cap \{ \text{first 70 diff.} \}
%  \mid
%  \text{first 2 diff.} ) \\
%  &=
%  \Pr(\text{first 3 diff.} \mid \text{first 2 diff.} ) \\
%  &\quad\cdot
%  \Pr(\{ \text{first 4 diff.} \}
%  \cap \dotsb \cap \{ \text{first 70 diff.} \}
%  \mid
%  \text{first 2 diff.} , \text{first 3 diff.} ).
%\end{align*}
%The reason this works is that conditional probability $\Pr(\cdot \mid C)$ is a
%probability function, so $\Pr(A \cap B \mid C) = \Pr(A \mid C) \Pr(B \mid A, C)$.
%Since
%\[
%  \{ \text{first 2 diff.} \cap \text{first 3 diff.} \}
%  =
%  \{ \text{first 3 diff.} \},
%\]
%the expression simplifies to
%\begin{align*}
%  \Pr(\{ \text{first 3 diff.} \}
%  &\cap \dotsb \cap \{ \text{first 70 diff.} \}
%  \mid
%  \text{first 2 diff.} ) \\
%  &=
%  \Pr( \text{first 3 diff.} \mid \text{first 2 diff.} ) \\
%  &\quad\cdot
%  \Pr(\{ \text{first 4 diff.} \}
%  \cap \dotsb \cap \{ \text{first 70 diff.} \}
%  \mid
%  \text{first 3 diff.} ).
%\end{align*}
%We can apply this trick repeatedly to get that
%

In class we showed how to use the multiplicative rule repeatedly to get:
\begin{align*}
\Pr(\text{all 50} &\text{ birthdays are different})
  = 
  \Pr(\text{first 2 diff.}) \cdot \Pr(\text{first 3 diff.} \mid \text{first 2 diff.}) \\
  &\cdot \Pr(\text{first 4 diff.} \mid \text{first 3 diff.})
  \cdot \Pr(\text{first 5 diff.} \mid \text{first 4 diff.})
  \cdot \dotsb
  \cdot \Pr(\text{first 50 diff.} \mid \text{first 49 diff.}).
\end{align*}
Using this expression we can compute
\[
\Pr(\text{all 50 birthdays are different})
  =
  \frac{364}{365}
  \cdot \frac{363}{365}
  \cdot \frac{362}{365}
  \cdot \dotsb
  \cdot \frac{365 - 49}{365}.
\]

We can do a similar calculation for other class sizes.
The following table shows the probabilities of having at least two students
with the same birthday for various class sizes:

\vspace{1\baselineskip}
\begin{center}
\begin{tabular}{cll}
\toprule
Class Size & $\Pr(\text{all diff.})$ & $\Pr(\text{at least 2 same})$ \\
\midrule
10 & 88\% & 12\% \\
20 & 59\% & 41\% \\
30 & 29\% & 71\% \\
40 & 11\% & 89\% \\
50 & $\phantom{0}$3\% & 97\% \\
60 & $\phantom{0}$0.6\% & 99.4\% \\
70 & $\phantom{0}$0.08\% & 99.92\% \\
\bottomrule
\end{tabular}
\end{center}
\vspace{1\baselineskip}

% birthday <- function(k) 1 - prod((365 - 0:(k - 1)) / 365)
% lnbirthday <- function(k) sum(log((365 - 0:(k - 1)) / 365))



\vspace{1\baselineskip}


\end{solution}


\end{questions}

\end{document}

