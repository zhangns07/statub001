\documentclass[answers,11pt]{exam}

\makeatletter
\newcommand{\course}[1]{\def\@course{#1}}
\makeatother

\makeatletter
\pagestyle{headandfoot}
\firstpageheader{}{\large \textbf{\@title \ifprintanswers ~-- Solutions \else \fi} \\ \@course}{}
\makeatother


\usepackage{amsmath}
\usepackage{amssymb}
\usepackage{booktabs}
\usepackage{graphicx}
\usepackage{subfig}
\usepackage{hyperref}

\DeclareGraphicsExtensions{.png,.pdf}


\DeclareMathOperator*{\Prob}{P}
\renewcommand{\Pr}{\Prob}
\DeclareMathOperator*{\E}{E}
\DeclareMathOperator*{\var}{var}
\DeclareMathOperator*{\sd}{sd}


\title{Confidence Intervals and Hypothesis Testing}
\course{STAT-UB.0001 -- Statistics for Business Control}

\begin{document}
\begin{questions}

\question Of the Stern MBA students who filled out an online class survey, $42$
reported their GMAT scores. The sample mean of the reported scores was $720$,
and the sample standard deviation was $35$.

\begin{parts}

\part What is a reasonable population to associate with this sample?

\begin{solution}
The GMAT scores of all Stern MBA students.
\end{solution}
\vspace{\stretch{1}}

\part What is the meaning of the ``population mean''?

\begin{solution}
$\mu$, the mean GMAT score of all Stern MBA students.
\end{solution}
\vspace{\stretch{1}}

\part Find a 95\% confidence interval for the population parameter.

\begin{solution}
We have $\bar x = 720$, $s = 35$, and $n = 42$.  For a 95\% confidence
interval, $\alpha  0.050$ and $\alpha/2 = 0.025$; with $n = 42$, there are $n
- 1 = 41$ degrees of freedom.  Consulting the $t$ table, we find
\[
  t_{\alpha/2,n-1} = t_{0.025,41} \approx 2.021.
\]
(Note that $\mathrm{df=41}$ is not in the table, so we look at
$\mathrm{df=40}$ instead.)
The 95\% confidence interval for $\mu$ is

\begin{align*}
  \bar x \pm t_{\alpha/2,n-1} \frac{s}{\sqrt{n}}
    &= 720 \pm 2.021 \cdot \frac{35}{\sqrt{42}} \\
    &= 720 \pm 10.9 \\
    &= (709.1, 730.9)
\end{align*}
\end{solution}
\vspace{\stretch{1}}

\part Under what conditions is the confidence interval valid?

\begin{solution}
Since $n$ is large ($n \geq 30$),
the interval is valid if we have a simple random sample.
\end{solution}
\vspace{\stretch{1}}

\end{parts}

\newpage


\question \label{ques:class-survey} Use the following sample means and sample
standard deviations of other questions from that class survey to form 95\% confidence intervals
for the population mean of each variable.

\begin{parts}

\part Dinners per month: $\bar x = 9.0$, $s = 4.6$, $n = 47$.

\begin{solution}
\begin{align*}
  t_{\alpha/2,n-1} &= t_{0.025,46} \approx 2.021
\end{align*}
\begin{align*}
  \bar x \pm t_{\alpha/2,n-1} \frac{s}{\sqrt{n}}
    &= 9.0 \pm 2.021 \cdot \frac{4.6}{\sqrt{47}} \\
    &= 9.0 \pm 1.36 \\
    &= (7.64, 10.36)
\end{align*}
\end{solution}
\vspace{\stretch{1}}

\part Age (years): $\bar x = 26.7$, $s = 6.1$, $n = 47$.

\begin{solution}
\begin{align*}
  t_{\alpha/2,n-1} &= t_{0.025,46} \approx 2.021
\end{align*}
\begin{align*}
  \bar x \pm t_{\alpha/2,n-1} \frac{s}{\sqrt{n}}
    &= 26.7 \pm 2.021 \cdot \frac{6.1}{\sqrt{47}} \\
    &= 26.7 \pm 1.80 \\
    &= (24.90, 28.50)
\end{align*}
\end{solution}
\vspace{\stretch{1}}


\part Time planned for studying per week (hours): $\bar x = 15.75$, $s = 10.5$, $n = 46$.

\begin{solution}
\begin{align*}
  t_{\alpha/2,n-1} &= t_{0.025,45} \approx 2.021
\end{align*}
\begin{align*}
  \bar x \pm t_{\alpha/2,n-1} \frac{s}{\sqrt{n}}
    &= 15.75 \pm 2.021 \cdot \frac{10.5}{\sqrt{46}} \\
    &= 15.75 \pm 3.13 \\
    &= (12.62, 18.88)
\end{align*}
\end{solution}
\vspace{\stretch{1}}

\end{parts}

\vspace{\stretch{1}}

\question In Problem~\ref{ques:class-survey}, what assumptions do we need for
the confidence intervals to be valid?  How could we check these assumptions?

\begin{solution}
We need that the sample is a simple random sample.  In this case, that is
equivalent to the sample being unbiased: every member of the population has an
equal chance of being selected.  If the sample is \emph{not} unbiased, then we
need the bias to be unrelated to the quantities of interest.  In particular,
we need that ``Dinners per Month,'' ``Age,'' and ``Study Time'' are
unrelated to the student's choice of statistics class.

We do \emph{not} need that the population is normal, because $n \geq 30$ in
each of these examples.
\end{solution}
\vspace{\stretch{1}}

\newpage


\question \label{ques:t-alpha} In each of the following situations, find
$\alpha$ and $t_{\alpha/2,n-1}$.

\begin{parts}

\part An 80\% confidence interval with $n = 10$.

\begin{solution}
\[
  \alpha = .20, \quad n - 1 = 9 \text{ degrees of freedom}, \quad t_{.100,9} = 1.383.
\]
\end{solution}

\vspace{\stretch{0.5}}

\part A 99\% confidence interval with $n = 25$.

\begin{solution}
\[
  \alpha = .01, \quad n - 1 = 24 \text{ degrees of freedom}, \quad t_{.005,24} = 2.797
\]
\end{solution}

\vspace{\stretch{0.5}}


\part A 90\% confidence interval with $n = 30$.

\begin{solution}
\[
  \alpha = .10, \quad n - 1 = 29 \text{ degrees of freedom}, \quad t_{.050,29}
\approx 1.701.
\]
Note that $\mathrm{df} = 29$ is not in the table, so we approximate the value
of $t_{.050,29}$ by using either $\mathrm{df} = 28$ or $\mathrm{df} = 30$
(either answer is acceptable).
\end{solution}

\vspace{\stretch{0.5}}

\end{parts}



\question \label{ques:confint-simple} A random sample of $36$ measurements was
selected from a population with unknown mean $\mu$.  The sample mean is
$\bar x = 12$ and the
sample standard deviation is $s = 18$.  Calculate an approximate
95\% confidence interval for $\mu$.  Use the approximation $t_{\alpha/2,n-1} =
t_{0.025,35} \approx 2$.


\begin{solution}
We compute a 95\% confidence interval for $\mu$ via the formula
\(
  \bar x \pm t_{0.025,n-1} \frac{s}{\sqrt{n}}.
\)
In this case, we get
\(
  12 \pm 2 \frac{18}{\sqrt{36}}
\)
i.e.,
\(
  12 \pm 6. 
\)
\end{solution}

\vspace{\stretch{1}}

\question Complete Problem~\ref{ques:confint-simple}, with a 99\% confidence interval
instead of a 95\% confidence interval.

\begin{solution}
For a $100 (1 - \alpha) \%$ confidence interval for $\mu$, we use the formula
\(
  \bar x \pm z_{\alpha/2} \frac{\sigma}{\sqrt{n}}.
\)
For a 99\% confidence interval, we have $\alpha = .01$ and $z_{\alpha/2} =
2.576$.  Thus, our confidence interval for $\mu$ is
\(
  12 \pm 2.576 \frac{18}{\sqrt{36}}
\)
i.e.,
\(
  12 \pm 7.728. 
\)

\end{solution}

\vspace{\stretch{1}}


\question Complete Problem~\ref{ques:confint-simple}, with an 80\% confidence interval
instead of a 95\% confidence interval.

\begin{solution}
For a $100 (1 - \alpha) \%$ confidence interval for $\mu$, we use the formula
\(
  \bar x \pm z_{\alpha/2} \frac{\sigma}{\sqrt{n}}.
\)
For a 80\% confidence interval, we have $\alpha = .20$ and $z_{\alpha/2} =
1.282$.  Thus, our confidence interval for $\mu$ is
\(
  12 \pm 1.282 \frac{18}{\sqrt{36}}
\)
i.e.,
\(
  12 \pm 3.846.
\)

\end{solution}

\vspace{\stretch{1}}


\newpage

\question How reliable is the SoHo Halal Guy's Yelp rating? The SoHo Halal Guy
at Broadway and Houston
(\url{http://www.yelp.com/biz/soho-halal-guy-new-york}) currently has 53 Yelp
reviews ($4$ 1-star; $1$ 2-star; $6$ 3-star; $17$ 4-star; and $25$ 5-star).
The average star rating is $4.1$ and the sample standard deviation of the
star ratings is $1.2$. How much should we trust the number ``4.1''?  We will
use a confidence interval to quantify the uncertainty associated with this
number.

\begin{parts}

\part What is a reasonable population to associate with this sample?

\begin{solution}
All ratings of the Halal Cart (past and future).
\end{solution}

\vspace{\stretch{1}}


\part What is the meaning of the population mean, $\mu$?

\begin{solution}
The parameter of interest is $\mu$, the mean start rating of all people who
ever review the Halal Cart.  Equivalently, the $\mu$ is equal to expected star
rating of a random Halal Cart reviewer.
\end{solution}

\vspace{\stretch{1}}


\part Find a 95\% confidence interval for the population mean, $\mu$.

\begin{solution}

For a $95$\% confidence interval, we have $\alpha = 0.05$ and $\alpha/2 =
0.025$.
The sample size is $n = 53$.  There are $n - 1 = 52$ degrees of freedom.  
Thus, using the $t$ table, we have
\[
  t_{\alpha/2, n-1} = t_{0.025,52} \approx 2.009.
\]
The 95\% confidence interval for the population mean, $\mu$, is
\begin{align*}
  \bar x \pm 2.009 \frac{s}{\sqrt{n}}
    &= 4.1 \pm 2.009 \frac{1.2}{\sqrt{53}} \\
    &= 4.1 \pm 0.33 \\
    &= (3.77, 4.43).
\end{align*}

\end{solution}

\vspace{\stretch{1}}

\part Under what conditions is the confidence interval valid?

\begin{solution}
For a confidence interval for a mean to be valid, we
need that (i) the observed sample is a simple random sample from the
population, and (ii) $n \geq 30$ or the population is normal.  Clearly,
assumption~(ii) holds. Here, it is
reasonable to assume (i) as long as the Halal Cart and its customer base do
not change in the future.
\end{solution}

\vspace{\stretch{1}}

\end{parts}


\question La Colombe at Lafayette and 4th St
(\url{http://www.yelp.com/biz/la-colombe-new-york-2/}) currently has $612$ Yelp
reviews ($16$ 1-star; $24$ 2-star; $50$ 3-star; $185$ 4-star; and $337$
5-star).  The average star rating is $4.31$ and the sample standard deviation
of the star ratings is $0.96$.  Find a 95\% confidence interval for the
expected rating of a random La~Colombe Yelp reviewer.

\begin{solution}
Since $n \geq 30$, we can approximate $t_{0.025,n-1} \approx 2$.  (A more
accurate approximation would be $t_{0.025,n-1} \approx 1.960$.  An approximate
95\% confidence interval for the population mean is
\begin{align*}
  \bar x \pm 2 \frac{s}{\sqrt{n}}
    &= 4.31 \pm 2 \frac{0.96}{\sqrt{612}} \\
    &= 4.31 \pm 0.08 \\
    &= (4.23, 4.39)
\end{align*}
\end{solution}

\vspace{\stretch{1}}


\newpage
\fullwidth{\section*{Confidence Interval for Proportion}}

\question A CNN/ORC post-debate poll surveyed 547 voters who watched the third
presidential debate on October 19, 2016. The results are at
\url{http://www.cnn.com/2016/10/19/politics/hillary-clinton-wins-third-presidential-debate-according-to-cnn-orc-poll}.
Of the respondents, 52\% thought that Clinton did the best job, while 39\%
thought that Trump did.

\begin{parts}

\part What is a reasonable population to associate with this sample?

\begin{solution}
The opinions of all voters who watched the debate.
\end{solution}

\vspace{\stretch{1}}


\part There are a few population parameters of interest. Choose one.

\begin{solution}
$p$, the proportion of all debate-watching voters who thought that
Clinton won. (Alternatively, the proportion of all debate-watching voters
who voters who thought the candidates tied, or that Trump won.)
\end{solution}

\vspace{\stretch{1}}


\part Find a 95\% confidence interval for the population parameter.

\begin{solution}
The sample proportion is $\hat p = 0.52$.  The sample size is $n = 547$.
For a 95\% confidence interval, we have $\alpha = 0.05$ and $\alpha/2 =
0.025$.  Thus,
\[
  z_{\alpha/2} = z_{0.050} = 1.960
\]
(use the $\mathrm{df} = \infty$ section of the $t$-table.)
Thus, the 95\% confidence interval for $p$ is
\begin{align*}
  \hat p \pm z_{\alpha/2} \sqrt{\frac{\hat p (1 - \hat p)}{n}}
    &= 0.52 \pm 1.96 \sqrt{\frac{(0.52) (1 - 0.52)}{547}} \\
    &= 0.52 \pm 0.04 \\
    &= (0.48, 0.56)
\end{align*}

For the proportion of debate-watching voters who thought that Trump won, the
confidence interval is
\begin{align*}
  \hat p \pm z_{\alpha/2} \sqrt{\frac{\hat p (1 - \hat p)}{n}}
    &= 0.39 \pm 1.96 \sqrt{\frac{(0.39) (1 - 0.39)}{547}} \\
    &= 0.39 \pm 0.04 \\
    &= (0.35, 0.43)
\end{align*}

\end{solution}

\vspace{\stretch{1}}


\part Under what conditions is the confidence interval valid?

\begin{solution}
We need a simple random sample, and we need to have expected at least 15
successes and 15 failures ($n p \geq 15$ and $n (1 - p) \geq 15$).  The latter
condition is almost certainly satisfied since we had $n \hat p = 284$
successes and $n (1 - \hat p) = 263$ failures in the sample.  For the former
condition, we need the sample to be unbiased.
\end{solution}

\vspace{\stretch{1}}


\end{parts}

\newpage



\question \label{ques:survey-prop} Use the following data from a Stern MBA class
survey to estimate the relevant population proportions. Give 95\% confidence
intervals for these proportions.

\begin{parts}

\part Gender: 17 Female, 30 Male.

\begin{solution}
If we let $p$ be the proportion of Female in the population, then we have
$\hat p = \frac{17}{17 + 30} = 0.36$ and $n = 17 + 30 = 47$.  The 95\%
confidence interval for $p$ is
\begin{align*}
  \hat p \pm z_{\alpha/2} \sqrt{\frac{\hat p (1 - \hat p)}{n}}
    &= 0.36 \pm 1.96 \sqrt{\frac{(0.36) (1 - 0.36)}{47}} \\
    &= 0.36 \pm 0.14 \\
    &= (0.22, 0.50)
\end{align*}
\end{solution}

\vspace{\stretch{1}}

\part Drinks at least one cup of coffee on a typical day: 37 Yes, 10 No.

\begin{solution}
If we let $p$ be the proportion of coffee drinkers in the population, then we have
$\hat p = \frac{37}{37 + 10} = 0.79$ and $n = 37 + 10  = 47$.  The 95\%
confidence interval for $p$ is
\begin{align*}
  \hat p \pm z_{\alpha/2} \sqrt{\frac{\hat p (1 - \hat p)}{n}}
    &= 0.79 \pm 1.96 \sqrt{\frac{(0.79) (1 - 0.79)}{47}} \\
    &= 0.79 \pm 0.12 \\
    &= (0.67, 0.91)
\end{align*}
\end{solution}

\vspace{\stretch{1}}

\part Political affiliation: 36 Democrat, 6 Republican, 5 Other. (For this
problem there are three different choices for the population parameter; choose
one of them.)

\begin{solution}
If we let $p$ be the proportion of the population that are Democrats. Then, we
have $\hat p = \frac{36}{36 + 6 + 5} = 0.77$ and $n = 36 + 6 + 5 = 47$.  The 95\%
confidence interval for $p$ is
\begin{align*}
  \hat p \pm z_{\alpha/2} \sqrt{\frac{\hat p (1 - \hat p)}{n}}
    &= 0.77 \pm 1.96 \sqrt{\frac{(0.77) (1 - 0.77)}{47}} \\
    &= 0.77 \pm 0.12 \\
    &= (0.65, 0.99)
\end{align*}

If instead we let $p$ be the proportion of the population that are
Republicans, then we have $\hat p = \frac{6}{36 + 6 + 5} = 0.13$ and $n = 36 + 6 + 5 = 47$.  The 95\%
confidence interval for $p$ is
\begin{align*}
  \hat p \pm z_{\alpha/2} \sqrt{\frac{\hat p (1 - \hat p)}{n}}
    &= 0.13 \pm 1.96 \sqrt{\frac{(0.13) (1 - 0.13)}{47}} \\
    &= 0.13 \pm 0.10 \\
    &= (0.03, 0.23)
\end{align*}

\end{solution}

\vspace{\stretch{1}}

\end{parts}

\question In Problem~\ref{ques:survey-prop}, what are the relevant
populations?

\begin{solution}
All Stern MBA students: their genders, whether they drink coffee, and 
their political affiliations.
\end{solution}

\vspace{\stretch{1}}

\question In Problem~\ref{ques:survey-prop}, what assumptions do we need for
the confidence intervals to be valid?

\begin{solution}
We need a simple random sample, and we need to have expected at least 15
successes and 15 failures in each sample. Since we do not have at least 15
successes and 15 failures in examples (b) and (c), these intervals are only
approximate, and the true confidence level may be below 95\%.
\end{solution}

\vspace{\stretch{1}}

\newpage 


\fullwidth{\section*{Hypothesis Test: Introduction}}

\question An analyst claims to have a reliable model for Yahoo's quarterly
revenues.  His model predicted that the most recent quarterly revenues could
be described as a normal random variable with mean \$1.5B and standard
deviation \$0.1B.  In actuality, the revenues were \$1.0B.  Is there evidence
of a problem with the analyst's model?  Why or why not?

\begin{solution}
The observed revenues were $(1.0 - 1.5) / (0.1) = 5$ standard deviations away
from the expected value predicted by the analyst's model.  If the analyst's
model were correct, the chance of observing a deviation this large or larger
would be extremely small ($0.00005733\%$).  This is evidence that there is a
problem with the analyst's model.
\end{solution}

\vspace{\stretch{1}}


\question David has a coin, which he claims to be fair (50\% chance of
``heads,'' and 50\% chance of ``tails'').  He flips the coin 10 times, and
gets ``heads'' all 10 times.  Do you believe him that the coin is fair?  Why
or why not?

\begin{solution}
If the coin were fair, there would be a $0.2\%$ chance of getting the same
outcome on all 10 flips.  That is, the observed string of 10 heads in a row
would be extremely unlikely.  This is evidence that the coin is not fair.
\end{solution}

\vspace{\stretch{1}}

\newpage 

\fullwidth{\section*{Test on a Population Mean}}

\question (Adapted from Stine and Foster, 4M~16.2).  Does stock in IBM return
a different amount on average than T-Bills?  We will attempt to answer this
question by using a dataset of the 264 monthly returns from IBM between 1990
and 2011.  Over this period, the mean of the monthly IBM returns was $1.26\%$
and the standard deviation was $8.27\%$.  We will take as given that the
expected monthly returns from investing in T-Bills is~$0.3\%$.

\begin{parts}

\part What is the sample?  What are the sample mean and standard deviation?

\begin{solution}
The $n = 264$ monthly IBM returns from 1990 to 2011.  The sample mean and
standard deviation (in \%) are
\begin{align*}
  \bar x &= 1.26 \\
  s &= 8.27
\end{align*}
\end{solution}

\vspace{\stretch{1}}


\part What is the relevant population?  What are the interpretations of
population mean and standard deviation?

\begin{solution}
All monthly IBM returns (past, present, and future).  The population mean,
$\mu$ represents the expected return for a month in the future.  The
population standard deviation, $\sigma$, represents the standard deviation of
the monthly returns for all months (past, present, and future).
\end{solution}

\vspace{\stretch{1}}


\part What are the null and alternative hypotheses for testing whether or not
IBM gives a different expected return from T-Bills (0.3\%)?

\begin{solution}
\begin{align*}
  H_0 &: \mu = 0.3 \\
  H_a &: \mu \neq 0.3
\end{align*}
\end{solution}

\vspace{\stretch{1}}

\newpage

\part Use an appropriate test statistic to summarize the evidence against the
null hypothesis.

\begin{solution}
If the null hypothesis were true ($\mu = 0.3$), then the sample mean would have been a
normal random variable with mean $\mu_{\bar X} = 0.3$ and standard deviation
$\sigma_{\bar X} = \sigma / \sqrt{n}$.  The test statistic
\[
  T = \frac{\bar X - \mu_0}{S/\sqrt{n}}
\]
would follow a $t$ distribution with $n-1=263$ degrees of freedom.  The
observed value of this statistic is
\[
  t = \frac{1.26 - 0.3}{8.27/\sqrt{264}} = 1.886
\]
\end{solution}

\vspace{\stretch{1}}


\part If the null hypothesis were true (there were no difference in expected
monthly returns between IBM and T-Bills) what would be the chance of observing data at
least as extreme as observed?

\begin{solution}
If we approximate the distribution of the test statistic under $H_0$ as a
standard normal random variable, then the chance of observing data at least as
extreme as observed would be
\[
  p = P(|Z| > 1.886) \approx 0.05743.
\]
\end{solution}

\vspace{\stretch{1}}

\part Is there compelling evidence (at significance level 5\%) of a difference
in expected monthly returns between IBM and T-Bills?

\begin{solution}
No, since $p \geq 0.05$, there is not compelling evidence.
\end{solution}

\vspace{\stretch{1}}

\part What assumptions do you need for the test to be valid?  Are these
assumptions plausible?

\begin{solution}
Since $n \geq 30$, we do not need to assume that the population is normal.
We need that the observed sample is a simple random sample from the
population; this might not hold if the period under observation (1990--2011)
is not typical for IBM.
\end{solution}

\vspace{\stretch{1}}

\end{parts}

\newpage

\fullwidth{\section*{Test Statistic and Observed Significance Level ($p$-value)}}

\question \label{ques:tstat} In each of the following examples, for the hypothesis test with
\begin{align*}
  H_0 &: \mu = \mu_0 \\
  H_a &: \mu \neq \mu_0
\end{align*}
find the test statistic ($t$) and the $p$-value.

\begin{parts}

\part $\mu_0 = 5$; $\bar x = 7$; $s = 10$; $n = 36$.

\begin{solution}
\begin{align*}
  t &= \frac{7 - 5}{10/\sqrt{36}} \\
    &= 1.2 \\
  p &\approx P(|Z| > 1.2) \\
    &= 0.2301
\end{align*}
\end{solution}

\vspace{\stretch{1}}


\part $\mu_0 = 90$; $\bar x = 50$; $s = 200$; $n = 64$.

\begin{solution}
\begin{align*}
  t &= \frac{50 - 90}{200/\sqrt{64}} \\
    &= -1.6 \\
  p &\approx P(|Z| > 1.6) \\
    &= 0.1096
\end{align*}
\end{solution}

\vspace{\stretch{1}}

\part $\mu_0 = 50$; $\bar x = 49.4$; $s = 2$; $n = 100$.

\begin{solution}
\begin{align*}
  t &= \frac{49.4 - 50}{2/\sqrt{100}} \\
    &= -3 \\
  p &\approx P(|Z| > 3) \\
    &= 0.002700
\end{align*}
\end{solution}

\vspace{\stretch{1}}


\end{parts}

\question For each example from problem~\ref{ques:tstat}:

\begin{parts}

\part Indicate whether a level 5\% test would reject $H_0$.

\begin{solution}
We reject $H_0$ if $p < 0.05$:
(a) do not reject $H_0$; (b) do not reject $H_0$; (c) reject $H_0$.
\end{solution}

\vspace{\stretch{1}}

\part Indicate whether a level 1\% test would reject $H_0$.

\begin{solution}
We reject $H_0$ if $p < 0.01$:
(a) do not reject $H_0$; (b) do not reject $H_0$; (c) reject $H_0$.
\end{solution}

\vspace{\stretch{1}}

\end{parts}


\end{questions}


\end{document}
