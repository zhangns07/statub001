\documentclass[answers,11pt]{exam}

\makeatletter
\newcommand{\course}[1]{\def\@course{#1}}
\makeatother
\newcommand{\ignore}[1]{}
\makeatletter
\pagestyle{headandfoot}
\firstpageheader{}{\large \textbf{\@title \ifprintanswers ~-- Solutions \else \fi} \\ \@course}{}
\makeatother



\usepackage{amsmath}
\usepackage{amssymb}
\usepackage{booktabs}
\usepackage{graphicx}
\usepackage{subfig}
\usepackage{enumitem}

\DeclareGraphicsExtensions{.png,.pdf}


\DeclareMathOperator*{\Prob}{P}
\renewcommand{\Pr}{\Prob}
\DeclareMathOperator*{\E}{E}
\DeclareMathOperator*{\var}{var}
\DeclareMathOperator*{\sd}{sd}


\title{Random Variables}
\course{STAT-UB.0001 -- Statistics for Business Control}

\begin{document}

\begin{questions}

\fullwidth{\section*{Permutations and Combinations}}
\question There are 5 messages on your answering machine, from A, B, C, D, and E, left in a random order.
\begin{parts}
\part What is the probability that they are in alphabetical order (ABCDE)? 
\begin{solution}
There are $5!$ number of ways to arrange 5 people, and ABCDE is one of them. So the probability is
\[
\frac{1}{\#\{\text{arrange five people}\}} = \frac{1}{5!}
\]
\end{solution}
\vspace{\stretch{0.5}}
\part What is the probability that A is immediately after D?
\begin{solution}
Treat ``DA'' as a single person F, and there are $4!$ number of ways to arrange the 4 people B, C, E, and F.
So the probability is
\[
  \frac{{\#\{\text{arrange five people, A is immediately after D}\}}}{{\#\{\text{arrange five people}\}}} = \frac{4!}{5}
\]
\end{solution}
\vspace{\stretch{0.5}}
\part If you only listen to the first three messages, what is the probability that these are from person A, B and C?
\begin{solution}
There are $\binom{5}{3}$ ways to chose a combination of three messages, and $\{A,B,C\}$ is one of the combinations. 
So the probability is 
\[
  \frac{1}{{\#\{\text{choose a combination of three people out of five}\}}} = \frac{1}{\binom{5}{3}}
\]
\end{solution}
\vspace{\stretch{0.5}}

\part If you only listen to the first three messages, what is the probability that E is among them?
\begin{solution}
In order to chose a combination of three people such at E is among them, the first step is to select E, and 
the second step is to chose two people out of the remaining 4. So the probability is
\[
  \frac{{\#\{\text{choose a combination of three people out of five, E is among them}\}}}
  {{\#\{\text{choose a combination of three people out of five}\}}} = \frac{1\cdot \binom{4}{2}}{\binom{5}{3}}
\]

\end{solution}
\vspace{\stretch{0.5}}
\end{parts}


%\question Suppose we pick 4 balls out of an urn with 12 red balls and 8 black
%balls.  What is the probability of $B = \text{``We get two balls of each
%color''}$?
%
%\begin{solution}
%The number of ways to pick 4 balls out of 20 is $\binom{20}{4}$.
%
%We use the multiplication rule as to count the number of ways to pick two
%balls of each color.  The sequence of experiments is (1) pick two red balls
%from the $12$, then (2) pick two black balls from the $8$.  There are
%$\binom{12}{2}$ ways to perform the first experiment, and $\binom{8}{2}$ to
%perform the second experiment, so there are $\binom{12}{2} \binom{8}{2}$
%total.
%
%Finally,
%\begin{align*}
%  P(B) &=
%    \frac{\#\{\text{picks with two balls of each color}\}}
%         {\#\{\text{picks with four balls\}}} \\
%       &= \frac{\binom{12}{2} \binom{8}{2}}{\binom{20}{4}}
%\end{align*}
%\end{solution}
%
%\vspace{\stretch{3}}
%
%\ifprintanswers\else\newpage\fi


\question \textbf{New York state lotto.}  You pick six of the numbers 1 through
54, and then in a televised drawing six of the numbers are selected.  If all
six of your numbers are selected then you win a share of the first place
prize.  If five or four of your numbers are selected you win a share of the
second or third prize.

\begin{parts}
\part How many ways are there to select 6 numbers for the lotto ticket?

\begin{solution}
\[
  \binom{54}{6}
  = \frac{54 \cdot 53 \cdot 52 \cdot 51 \cdot 50 \cdot 49}
         {6 \cdot 5 \cdot 4 \cdot 3 \cdot 2 \cdot 1}
  = 25827165
\]
\end{solution}

\vspace{\stretch{1}}


\part How many ways are there to select a first prize number?

\begin{solution}
\[
  1
\]
\end{solution}

\vspace{\stretch{1}}


\part What is the probability of selecting a first prize number?


\begin{solution}
\begin{align*}
   P(\text{first prize})
     &= \frac{\#\{\text{lotto tickets that match all six numbers}\}}
         {\#\{\text{lotto tickets\}}} \\
     &= \frac{1}{\binom{54}{6}} \\
     &= 1 / 25827165 \\
     &= 0.000004\%
\end{align*}
\end{solution}
\vspace{\stretch{1}}

\part What is the probability of selecting a second prize number?
\begin{solution}
How many lotto tickets have five matches and one mismatch from our pick?
Consider the following sequence of experiments for enumerating all such
possibilities: (1) pick $5$ numbers out of our six to match; (2) pick $1$
number to mismatch.  There are $\binom{6}{5}$ possible outcomes for experiment
(1), and $\binom{48}{1}$ possible outcomes for experiment (2).  (Note that $48
= 54 - 6$; there are $48$ numbers in the range 1--54 which aren't on our lotto
ticket.)  Thus,
\begin{align*}
   P(\text{second prize})
    &= \frac{\#\{\text{lotto tickets that match exactly five numbers}\}}
         {\#\{\text{lotto tickets\}}} \\
     &= \frac{\binom{6}{5}\binom{48}{1}}{\binom{54}{6}}.
\end{align*}

The computation for third prize is similar to the computation for second
prize:

\begin{align*}
   P(\text{third prize})
    &= \frac{\#\{\text{lotto tickets that match exactly four numbers}\}}
         {\#\{\text{lotto tickets\}}} \\
     &= \frac{\binom{6}{4}\binom{48}{2}}{\binom{54}{6}}.
\end{align*}
\end{solution}


\end{parts}


\vspace{\stretch{1}}
\newpage

\question \textbf{Quality assurance.}  Suppose we have a batch of $100$ light bulbs,
which contains $5$ defective bulbs.  If we pick $10$ for testing, what is the
probability that no bulbs in the sample are defective?  We can answer this
question in three steps.

\begin{parts}

\part How many ways are there of picking $10$ bulbs for testing out of $100$?

\begin{solution}
\[
  \binom{100}{10}
\]
\end{solution}

\vspace{\stretch{0.3}}

\part How many ways are there of picking $10$ non-defective bulbs?

\begin{solution}
\[
  \binom{95}{10}
\]
\end{solution}

\vspace{\stretch{0.3}}

\part What is the probability that there are no defective bulbs in your sample
of $10$?

\begin{solution}
\begin{align*}
  P(\text{no defects in sample})
    &= \frac{\binom{95}{10}}{\binom{100}{10}} \\
    &= 58\%.
\end{align*}
\end{solution}

\vspace{\stretch{0.3}}

\end{parts}


\question \textbf{The Birthday Problem.} A class has 20 students.  What is the probability that at least
two students have the same birthday?  Assume that each person in the class was
assigned a random birthday between January 1 and December 31.

\begin{solution}
Assume that everyone in the class is randomly assigned a birthday, which
corresponds to number between 1 and 365 representing the day of the year.
It turns out to be much easier to compute the probability using the complement
rule, as
\[
  \Pr(\text{at least 2 people have the same birthday})
  = 1 - \Pr(\text{all 20 birthdays are different}).
\]

We use counting rules to compute the probability that all 20 birthdays are different. 
We first compute the size of sample space. That is $365^{20}$.

Next, we compute the number of ways to assign 365 birthdays to 20 students, such that no two students can have the same birthday.
That is $P(365,20)=\frac{365!}{345!}$.

Thus the probability of all 20 birthdays are different is
$$\frac{P(365,20)}{365^{20}} = \frac{365}{365}\cdot \frac{364}{365}\cdot\frac{363}{365}\cdots \frac{365-19}{365}.$$

\end{solution}

\vspace{\stretch{1}}

\newpage
\fullwidth{\section*{Probability Distribution Function and Expectation}}

\question \label{ques:rv-coinflip-game} Consider the following game:
\begin{enumerate}
\item You pay \$6 to flip a coin.
\item If the coin lands heads, you get \$10; otherwise, you get nothing.
\end{enumerate}

\begin{parts}

\part Would you play this game?  Why or why not?

\begin{solution}
It will usually be beneficial to play the game when our expected winnings are
positive.  In this situation, if we play the game many times, then in the long
run we will make a profit.
\end{solution}

\vspace{\stretch{1}}


\part What is the random experiment involved in the game?  What are the sample
space?  What are the probabilities of the sample points?

\begin{solution}
The random experiment is the coin flip.  
The sample points for  the coin flip are H and T; each of
these has probability $\frac{1}{2}$.  
\end{solution}

\vspace{\stretch{1}}


\part \label{part:rv-coinflip-game-pdf} Let $W$ be the random variable equal
to the amount of money you win from playing the game.  If you lose money, $W$
will be negative.  Find the value of $W$ for each of the sample points.

\begin{solution}

The values of the random variable
corresponding to the sample points are as follow:

\begin{center}
\begin{tabular}{cr}
\toprule
Outcome & $W$ \\
\midrule
H & 4 \\
T & -6 \\
\bottomrule \\
\end{tabular}
\end{center}

\end{solution}

\vspace{\stretch{1}}


\part Describe $W$ in terms of its probability distribution function (PDF).

\begin{solution}
The PDF of $W$ is given by the table:

\begin{center}
\begin{tabular}{c|rr}
$w$ & -6 & 4 \\
\hline
$p(w)$ & 0.5 & 0.5 
\end{tabular}
\end{center}
\end{solution}

\vspace{\stretch{1}}


\part \label{part:rv-coinflip-game-expect} What are your expected winnings?
That is, what is $\mu$, the expectation of $W$?

\begin{solution}

Using the PDF computed in part~(\ref{part:rv-coinflip-game-pdf}), the expected value
of $W$ is
\begin{align*}
  \mu
  &= (.5)(-6) + (.5)(4) \\
  &= -1.
\end{align*}
On averate, we lose \$1 every time we play the game.
\end{solution}

\vspace{\stretch{1}}

\end{parts}


\newpage

\question \label{ques:rv-2coins-outcomes}
Suppose you flip two coins.  Let $X$ be the random variable which counts
the number of heads on the two tosses.  

\begin{parts}

\part List all of the sample points of the experiment, along with the
corresponding values of $X$.

\begin{solution}
\begin{center}
\begin{tabular}{cr}
\toprule
Outcome & $X$ \\
\midrule
HH & 2 \\
HT & 1 \\
TH & 1 \\
TT & 0 \\
\bottomrule
\end{tabular}
\end{center}
\end{solution}

\vspace{\stretch{1}}

\part \label{part:rv-2coins-pdf} Compute the probability distribution function of $X$.

\begin{solution}
\begin{center}
\begin{tabular}{c|ccc}
$x$ & 0 & 1 & 2 \\
\hline
$p(x)$
& $\frac{1}{4}$ 
& $\frac{2}{4}$ 
& $\frac{1}{4}$ 
\end{tabular}
\end{center}
\end{solution}

\vspace{\stretch{1}}


\part Compute the expectation of $X$.

\begin{solution}
Using the PDF we computed in part~(\ref{part:rv-2coins-pdf}), the expectation
of $X$ is
\begin{align*}
\E(X)
  &= (\tfrac{1}{4}) (0) + (\tfrac{2}{4}) (1) + (\tfrac{1}{4}) (2) \\
  &= 1.
\end{align*}
\end{solution}
\vspace{\stretch{1}}


\part What is the interpretation of the expectation of $X$?

\begin{solution}
If we were to repeat the experiment many times, getting a different value of
$X$ each time, then the average value of $X$ will be close to $1$.
\end{solution}

\vspace{\stretch{1}}

\end{parts}



\question 
\label{ques:rv-coffee}
Let $X$ be a random variable describing the number of cups of coffee a
randomly-chosen member of the class drinks on a typical day.  There is a
22\% chance that the student has one cup, a 16\% chance that the student has
two cups, a 16\% chance that the student has three cups, an 11\% chance that
the student has four cups, and a 3\% chance that the student has five cups.
Also, there is a 32\% chance that the student doesn't drink any coffee.

\begin{parts}

\part Let $p(x)$ be the probability distribution function of $X$.  Fill in the
following table:
\begin{center}
\begin{tabular}{c|cccccc}
  $x$ & 0 & 1 & 2 & 3 & 4 & 5 \\
\hline
$p(x)$ & \phantom{.32} & \phantom{.22} & \phantom{.16} & \phantom{.16} & \phantom{.11} & \phantom{.03}
\end{tabular}
\end{center}

\begin{solution}
\begin{center}
\begin{tabular}{c|cccccc}
  $x$ & 0 & 1 & 2 & 3 & 4 & 5 \\
\hline
$p(x)$ & .32 & .22 & .16 & .16 & .11 & .03
\end{tabular}
\end{center}
\end{solution}

\vspace{2\baselineskip}

\part Find $\E(X)$, the expectation of $X$.

\begin{solution}
\begin{align*}
  \E(X)
  &= (.32) (0) + (.22) (1) + (.16) (2) + (.16) (3) + (.11) (4) + (.03) (5)
  &= 1.61.
\end{align*}
\end{solution}

\vspace{\stretch{1}}

\part What is the interpretation of the expectation of $X$?

\begin{solution}
The long-run sample mean.  If we performed the random experiment upon which
$X$ is measured many times, getting a different value of $X$ each time, then
the sample mean would be very close to the expectation of $X$.
In particular, if every day we sample a student from the class and measure how
many cups of coffee they drink, then after awhile, the average number of cups
of coffee will be close to the expectation (1.61).
\end{solution}

\vspace{\stretch{1}}

\end{parts}


\newpage
\fullwidth{\section*{Variance and Standard Deviation}}

\question This is a continuation of problem~\ref{ques:rv-coffee}.

\begin{parts}

\part Find $\var(X)$ and $\sd(X)$, the variance and standard deviation of $X$,
the number of cups of coffee that a random student from the class drinks on a
typical day.


\begin{solution}
\begin{align*}
\begin{split}
  \var(X)
  &= (.32) (0-1.61)^2 + (.22) (1-1.61)^2 + (.16) (2-1.61)^2 + (.16) (3-1.61)^2
  \\
  &\quad+ (.11) (4-1.61)^2 + (.03) (5-1.61)^2
\end{split} \\
  &= 2.2179
\end{align*}

The standard deviation of $X$ is given by
\[
  \sd(X) = \sqrt{\var(X)} = \sqrt{2.2179} = 1.48.
\]
\end{solution}

\vspace{\stretch{1}}

\part What is the interpretation of the standard deviation of $X$?

\begin{solution}
The long-run sample standard deviation.  If we performed the random experiment upon which
$X$ is measured many times, getting a different value of $X$ each time, then
the sample standard deviation would be very close to the standard deviation of
$X$.  
In particular, if every day we sample a student from the class and measure how
many cups of coffee they drink, then after awhile, the sample standard
deviation of the number of cups of coffee will be close to the 1.48.
\end{solution}

\vspace{\stretch{1}}


\end{parts}



\question Consider the following game:
\begin{enumerate}
\item You pay \$6 to pick a card from a standard 52-card deck.
\item If the card is a diamond ($\diamondsuit$), you get \$22; if the card is a
heart ($\heartsuit$), you get \$6; otherwise, you get nothing.
\end{enumerate}
Perform the following calculations to decide whether or not you would play
this game.

\begin{parts}

\part \label{part:rv-cardsuit-game-pdf} Let $W$ be the random variable equal
to the amount of money you win from playing the game.  If you lose money, $W$
will be negative.  Find the PDF of $W$.

\begin{solution}
The sample points corresponding to the suit of the card are $\spadesuit$,
$\heartsuit$, $\clubsuit$, and $\diamondsuit$; each of these has probability
$\tfrac{1}{4}$.  The values of the random variable $W$
corresponding to the sample points are as follow:

\begin{center}
\begin{tabular}{cr}
\toprule
Outcome & $W$ \\
\midrule
$\spadesuit$ & -6 \\
$\heartsuit$ & 0 \\
$\clubsuit$ & -6 \\
$\diamondsuit$ & 16 \\
\bottomrule \\
\end{tabular}
\end{center}

Thus, the PDF of $W$ is given by the table:

\begin{center}
\begin{tabular}{c|rrr}
$w$ & -6 & 0 & 16 \\
\hline
$p(w)$ & 0.50 & 0.25  & 0.25
\end{tabular}
\end{center}
\end{solution}

\vspace{\stretch{1}}


\part \label{part:rv-cardsuit-game-expect} What are your expected winnings?
That is, what is $\mu$, the expectation of $W$?

\begin{solution}

Using the PDF computed in part~(\ref{part:rv-cardsuit-game-pdf}), the expected value
of $W$ is
\begin{align*}
  \mu
  &= (.50)(-6) + (.25)(0) + (.25)(16) \\
  &= 1.
\end{align*}
On average, we win \$1 every time we play the game.
\end{solution}

\vspace{\stretch{1}}


\part What is the standard deviation of $W$?

\begin{solution}

Using the PDF computed in part~(\ref{part:rv-cardsuit-game-pdf}), and the expected value
computed in part~(\ref{part:rv-cardsuit-game-expect}), we compute the variance
of $W$ as
\begin{align*}
  \sigma^2
  &= (.50)(-6 - 1)^2 + (.25)(0 - 1)^2 + (.25)(16 - 1)^2 \\
  &= 81.
\end{align*}
Thus, the standard deviation of $W$ is
\[
  \sigma = \sqrt{81} = 9.
\]
\end{solution}

\vspace{\stretch{1}}


\part What are the interpretations of the expectation and standard deviation
of $W$?

\begin{solution}
If we played the game many many times, then the average of our winnings over
all times we played would be close to the \$1, and the standard
deviations of our winnings over all times we played would be close to \$9.
\end{solution}

\vspace{\stretch{1}}


\end{parts}

\newpage

\fullwidth{\section*{Properties of Expectation}}

\question \textbf{Affine Transformations.}
Let $X$ be a random variable with expectation $\mu_X = 2$.  What is the
expectation of $5 X + 2$?

\begin{solution}
\[
  5 \mu_X + 2 = 12.
\]
\end{solution}
\vspace{\stretch{1}}


\question \textbf{Sums of Random Variables.}
Let $X$ and $Y$ be random variables with $\mu_X = 1$, 
$\mu_Y = -5$.  What is $\E(X + Y)$?

\begin{solution}
\[
  \E(X + Y) = \mu_X + \mu_Y = 1 + (-5) = -4.
\]
\end{solution}

\vspace{\stretch{1}}


\question Let $X$ and $Y$ be random variables with $\mu_X = -2$, $\mu_Y = 3$.

\begin{parts}

\part Find the expectation of $-3X + 2$.

\begin{solution}
\[
  \E(-3X + 2) = -3 \mu_X + 2 = -3 (-2) + 2 = 8.
\]
\end{solution}

\vspace{\stretch{1}}

\part Find the expectation of $X + Y$.

\begin{solution}
\[
  \E(X + Y) = \mu_X + \mu_Y = 1.
\]
\end{solution}

\vspace{\stretch{1}}

\end{parts}


\question You invite four people to go out to dinner on Friday night.  The
attendance probabilities for the four potential guests are 50\%, 20\%, 30\%, and 90\%.

\begin{parts}
\part Find the expected number of guests.

\begin{solution}
Let $X$ be the number of guests.  Then, $X$ can be written as
\[
  X = Y_1 + Y_2 + Y_3 + Y_4,
\]
where
\[
  Y_i
  =
  \begin{cases}
    1 &\text{if guest $i$ attends,} \\
    0 &\text{otherwise.}
  \end{cases}
\]
Then, $\E[Y_1] = .50$, $\E[Y_2] = .20$, $\E[Y_3 = .30]$, and $\E[Y_4] = .90$,
so
\begin{align*}
  \E[X]
    &= \E[Y_1 + Y_2 + Y_3 + Y_4] \\
    &= \E[Y_1] + \E[Y_2] + \E[Y_3] + \E[Y_4] \\
    &= .50 + .20 + .30 + .90 \\
    &= 1.9.
\end{align*}
\end{solution}

\vspace{\stretch{1}}


\part \label{part:expected-dinner-cost} The dinner will be a \emph{prix fixe}
meal, costing \$50 per person.  What is the expected total cost for yourself
and your guests?

\begin{solution}
The total cost is $C = 50 + 50 X$.  Thus,
\begin{align*}
  \E[C] &= \E[50 + 50 X] \\
        &= 50 + 50 \E[X] \\
        &= 50 + 50 (1.9) \\
        &= 145.
\end{align*}
The expected total cost is \$145.
\end{solution}

\vspace{\stretch{1}}


\part What is the interpretation of your answer to part~(\ref{part:expected-dinner-cost})?

\begin{solution}
If there were many similar nights with the same circumstances, then the
average cost of all of the dinners would be \$145.
\end{solution}

\vspace{\stretch{1}}

\end{parts}



\end{questions}

\end{document}

