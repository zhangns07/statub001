%%%%%%%%%%%%%%%%%%%%%%%%%%%%%%%%%%%%%%%%%%%%%%%
%%%This is a science homework template. Modify the preamble to suit your needs. 
%The junk text is   there for you to immediately see how the headers/footers look at first 
%typesetting.


\documentclass[11pt]{article}

%AMS-TeX packages
\usepackage{amssymb,amsmath,amsthm}
%geometry (sets margin) and other useful packages
\usepackage{alltt}
\usepackage{booktabs}
\usepackage{enumerate}
\usepackage{fullpage}
\usepackage{graphicx,ctable,booktabs}
\usepackage{hyperref}


%
%Redefining sections as problems
%
\makeatletter
\newenvironment{problem}{\@startsection
       {section}
       {1}
       {-.2em}
       {-3.5ex plus -1ex minus -.2ex}
       {2.3ex plus .2ex}
       {\pagebreak[3]%forces pagebreak when space is small; use \eject for better results
       \large\bf\noindent{Problem }
       }
       }
       {%\vspace{1ex}\begin{center} \rule{0.3\linewidth}{.3pt}\end{center}}
       \begin{center}\large\bf \ldots\ldots\ldots\end{center}}
\makeatother



\DeclareMathOperator*{\sd}{sd}


%
%Fancy-header package to modify header/page numbering 
%
%\usepackage{fancyhdr}
%\pagestyle{fancy}
%\addtolength{\headwidth}{\marginparsep} %these change header-rule width
%\addtolength{\headwidth}{\marginparwidth}
%\lhead{Problem \thesection}
%\chead{} 
%\rhead{\thepage} 
%\lfoot{\small\scshape course name}
%\cfoot{} 
%\rfoot{\footnotesize HW \#1}
%\renewcommand{\headrulewidth}{.3pt}
%\renewcommand{\footrulewidth}{.3pt}
%\setlength\voffset{-0.25in}
%\setlength\textheight{648pt}

%%%%%%%%%%%%%%%%%%%%%%%%%%%%%%%%%%%%%%%%%%%%%%%

%
%Contents of problem set
%    
\begin{document}

\begin{center}
  \large
  \textbf{Homework \#3 -- Due Thursday, Jul.~26} \\
  STAT-UB.0001 -- Statistics for Business Control \\
\end{center}


\thispagestyle{empty}

\begin{problem}{}

A multiple-choice quiz has 20 questions. Each question has five possible
answers, of which only one is correct.

\begin{enumerate}[(a)]

\item What is the probability that sheer guesswork will yield exactly 10 correct answers?

\item What is the expected number of correct answers by sheer guesswork?

\item Suppose 5 points are awarded for a correctly answered question. How many
points should be deducted for an incorrectly answered question, so that for a
student guessing randomly, the expected score on a question is zero? (Most
standardized tests use this method to set penalties for guessing.)

\item If a student is able to correctly eliminate one option as a possible
correct answer but is still guessing randomly, what happens to his/her
expected score for that question? Use your answer to (c) as the number of
points being deducted for an incorrect answer.

\end{enumerate}

\end{problem}

\begin{problem}{}

A Motel has 16 bedrooms. From past experience, the manager knows that 20\% of
the people who make room reservations don't show up. The manager accepts 20
reservations. If a customer with a reservation shows up and the motel has run
out of rooms, it is the motel's policy to pay \$100 as compensation to the
customer. 
\begin{enumerate}[(a)]
\item What is the expected number of customers that will show up? 

\item What is the expected value of the compensation that the motel must pay?

\end{enumerate}
\end{problem}


\begin{problem}{}
An automatic car wash takes exactly 5 minutes to wash a car. On average, 8 cars per hour arrive at the car wash.
Suppose the number of cars arrive follows a Poisson distribution. Now suppose 15 minutes before closing time, 3 cars are in line.
If the car wash is in continuous use until closing time, what is the chance that anyone will be in line at closing time?
\end{problem}

\begin{problem}{}
 
Suppose that you throw two dice. Each die can come up as $1$, $2$, $3$, $4$,
$5$ or $6$, and the results from the two dice are independent of each other.
We are interested in the random variable $X$, the sum of the two numbers that
land face up. The possible values for $X$ are $2, 3, \dotsc, 12$. 
 
\begin{enumerate}[(a)]

\item Make a table giving the probability distribution of $X$.  Explain
briefly how you did the calculations. 

\item Show that $E(X) = 7$ and $\mathrm{var}(X) = 210/36 = 5.833$

\item The distribution of $X$ would look somewhat bell-shaped. (This is not a coincidence. The more
dice you toss, the closer the distribution of the sum comes to a normal
distribution. More on this later in the course.) For now, let's see how well
the empirical rule works. Show that the probability that $X$ is within 
$E(X) \pm \mathrm{sd}(X)$ is $24/36=0.667$.  
Show that the probability that $X$ is within $E(X) \pm 2\mathrm{sd}(X)$ is $34/36=0.944$. 

\end{enumerate}

\end{problem}




\end{document}


