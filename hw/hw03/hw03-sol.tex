%%%%%%%%%%%%%%%%%%%%%%%%%%%%%%%%%%%%%%%%%%%%%%%
%%%This is a science homework template. Modify the preamble to suit your needs. 
%The junk text is   there for you to immediately see how the headers/footers look at first 
%typesetting.


\documentclass[answers,11pt]{exam}

%AMS-TeX packages
\usepackage{amssymb,amsmath,amsthm}
%geometry (sets margin) and other useful packages
\usepackage{alltt}
\usepackage{booktabs}
\usepackage{enumerate}
\usepackage{fullpage}
\usepackage{graphicx,ctable,booktabs}
\usepackage{hyperref}

\DeclareMathOperator*{\Prob}{P}
\renewcommand{\Pr}{\Prob}
\DeclareMathOperator*{\E}{E}
\DeclareMathOperator*{\var}{var}
\DeclareMathOperator*{\sd}{sd}
%\renewcommand{\binom}[2]{\ensuremath{{}_{#1}C_{#2}}\,}

%
%Redefining sections as problems
%
\makeatletter
\newenvironment{problem}{\@startsection
       {section}
       {1}
       {-.2em}
       {-3.5ex plus -1ex minus -.2ex}
       {2.3ex plus .2ex}
       {\pagebreak[3]%forces pagebreak when space is small; use \eject for better results
       \large\bf\noindent{Problem }
       }
       }
       {%\vspace{1ex}\begin{center} \rule{0.3\linewidth}{.3pt}\end{center}}
       \begin{center}\large\bf \ldots\ldots\ldots\end{center}}
\makeatother




%
%Fancy-header package to modify header/page numbering 
%
%\usepackage{fancyhdr}
%\pagestyle{fancy}
%\addtolength{\headwidth}{\marginparsep} %these change header-rule width
%\addtolength{\headwidth}{\marginparwidth}
%\lhead{Problem \thesection}
%\chead{} 
%\rhead{\thepage} 
%\lfoot{\small\scshape course name}
%\cfoot{} 
%\rfoot{\footnotesize HW \#1}
%\renewcommand{\headrulewidth}{.3pt}
%\renewcommand{\footrulewidth}{.3pt}
%\setlength\voffset{-0.25in}
%\setlength\textheight{648pt}

%%%%%%%%%%%%%%%%%%%%%%%%%%%%%%%%%%%%%%%%%%%%%%%

%
%Contents of problem set
%    
\begin{document}

\begin{center}
  \large
  \textbf{Homework \#3 -- Due Thursday, Jul.~26} \\
  STAT-UB.0001 -- Statistics for Business Control \\
\end{center}


\thispagestyle{empty}

\begin{problem}{}

A multiple-choice quiz has 20 questions. Each question has five possible
answers, of which only one is correct.

\begin{enumerate}[(a)]

\item What is the probability that sheer guesswork will yield exactly 10 correct answers?
\begin{solution}
The number of correct answers, $X$, is binomial with $n = 20$ and $p = 1/5 =0.2$.
\begin{align*}
P(X = 10)
   &=  \binom{20}{10} (.20)^{10} (.80)^{10} = 0.00203
\end{align*}


\end{solution}

\item What is the expected number of correct answers by sheer guesswork?

\begin{solution}
\[
  E[X] = (20) (.2) = 4.
\]
\end{solution}

\item Suppose 5 points are awarded for a correctly answered question. How many
points should be deducted for an incorrectly answered question, so that for a
student guessing randomly, the expected score on a question is zero? (Most
standardized tests use this method to set penalties for guessing.)
\begin{solution}
If we let $c$ denote the cost for guessing incorrectly, and if $Y$ denotes the
score for a question on which a student guesses randomly, then $Y$
has the PDF:
\begin{center}
\begin{tabular}{c|cc}
$y$    & $-c$   & $5$ \\
\hline
$p(y)$ & $0.80$ & $0.20$
\end{tabular}
\end{center}
For the expected score to be zero, we must have
\[
  \E[Y] = (0.80)(-c) + (0.20)(5) = 0,
\]
so that
\[
    c = \frac{(0.20)(5)}{(0.80)}
      = 1.25.
\]

\end{solution}

\item If a student is able to correctly eliminate one option as a possible
correct answer but is still guessing randomly, what happens to his/her
expected score for that question? Use your answer to (c) as the number of
points being deducted for an incorrect answer.
\begin{solution}
In this case $Y$ has the PDF
\begin{center}
\begin{tabular}{c|cc}
$y$    & $-1.25$   & $5$ \\
\hline
$p(y)$ & $0.75$ & $0.25$
\end{tabular}
\end{center}
The expected score is
\[
  \E[Y] = (0.75)(-1.25) + (0.25)(5)
        = 0.3125.
\]


\end{solution}

\end{enumerate}

\end{problem}

\newpage
\begin{problem}{}

A Motel has 16 bedrooms. From past experience, the manager knows that 20\% of
the people who make room reservations don't show up. The manager accepts 20
reservations. If a customer with a reservation shows up and the motel has run
out of rooms, it is the motel's policy to pay \$100 as compensation to the
customer. 
\begin{enumerate}[(a)]
\item What is the expected number of customers that will show up? 
\begin{solution}
Let $X$ denote the number of customers that show up.  This is binomial with $n
= 20$ and $p = 0.80$.  

We can compute
\[
\E [X] = (20)(0.80) = 16.
\]
\end{solution}

\item What is the expected value of the compensation that the motel must pay?
\begin{solution}
The manager will have to pay compensation if $X > 16$.
We can compute
\begin{align*}
  P(X = 17) &= \binom{20}{17} (.80)^{17} (.20)^3 = 0.2054\\
  P(X = 18) &= \binom{20}{18} (.80)^{18} (.20)^2 = 0.1369\\
  P(X = 19) &= \binom{20}{19} (.80)^{19} (.20)^1 = 0.0576\\
  P(X = 20) &= \binom{20}{20} (.80)^{20} (.20)^0 = 0.0115\\
\end{align*}
The probability that the manager will not pay compensation is
\begin{align*}
  P(X \leq 16)
    &= 1 - (0.2054 + 0.1369 + 0.0576 + 0.0115)
    = 0.5886.
\end{align*}

Let $Y$ denote the amount of compensation that the
manager has to pay.  We can use the probabilities above to compute the PDF of
$Y$:

\begin{center}
\begin{tabular}{c|ccccc}
$x$    &      $\leq$ 16 &    17&    18&    19&    20\\\hline
$y$    &      0 &    100 &    200 &    300 &    400  \\
\hline
$p(y)$ & 0.5886 & 0.2054 & 0.1369 & 0.0576 & 0.0115 
\end{tabular}
\end{center}

The expected value of $Y$ is
\begin{align*}
  \E[Y] &= 
    (0.5886)(0)
    + (0.2054)(100)
    + (0.1369)(200)
    + (0.0576)(300)
    + (0.0115)(400)\\
    &= 69.8.
\end{align*}
The expected compensation is \$69.8.


\end{solution}

\end{enumerate}
\end{problem}


\begin{problem}{}
An automatic car wash takes exactly 5 minutes to wash a car. On average, 8 cars per hour arrive at the car wash.
Suppose the number of cars arrive follows a Poisson distribution. Now suppose 15 minutes before closing time, 3 cars are in line.
If the car wash is in continuous use until closing time, what is the chance that anyone will be in line at closing time?
\begin{solution}
Let $X$ denote the number of cars that arrive in the 15 mins. This is a Poisson with $\lambda=\frac{(8)(15)}{(60)}=2$.
If one or more cars arrive at the car wash within the 15 mins, then there will be someone in line at closing time. Thus,
we can compute
\begin{align*}
\Pr(X > 0) & = 1 - \Pr(X=0) \\
                 &= 1- \frac{\lambda^0 e^{-\lambda}}{0!} \\
                 &= 1- \frac{(1)( e^{-2})}{1} \\
                 &= 0.864.
\end{align*}
\end{solution}
\end{problem}

\newpage
\begin{problem}{}
 
Suppose that you throw two dice. Each die can come up as $1$, $2$, $3$, $4$,
$5$ or $6$, and the results from the two dice are independent of each other.
We are interested in the random variable $X$, the sum of the two numbers that
land face up. The possible values for $X$ are $2, 3, \dotsc, 12$. 
 
\begin{enumerate}[(a)]

\item Make a table giving the probability distribution of $X$.  Explain
briefly how you did the calculations. 
\begin{solution}
The first and second rolls are each equally likely to be any of the numbers
from 1--6.  We can compute a table of the values of $X$ for each pair
$(\text{first roll}, \text{second roll})$:

\begin{center}
\begin{tabular}{c|cccccc}
  & 1 & 2 & 3 &  4 &  5 &  6 \\
\hline
1 & 2 & 3 & 4 &  5 &  6 &  7 \\
2 & 3 & 4 & 5 &  6 &  7 &  8 \\
3 & 4 & 5 & 6 &  7 &  8 &  9 \\
4 & 5 & 6 & 7 &  8 &  9 & 10 \\
5 & 6 & 7 & 8 &  9 & 10 & 11 \\
6 & 7 & 8 & 9 & 10 & 11 & 12 \\
\end{tabular}
\end{center}

Each of the 36 sample points is equally likely.  There is one sample point
with $X = 2$, two sample points with $X = 3$, etc.  Thus, we have the pdf

\begin{center}
\begin{tabular}{c|ccccccccccc}
$x$ & 2 & 3 & 4 & 5 & 6 & 7 & 8 & 9 & 10 & 11 & 12 \\
\hline
$p(x)$
& $\frac{1}{36}$
& $\frac{2}{36}$
& $\frac{3}{36}$
& $\frac{4}{36}$
& $\frac{5}{36}$
& $\frac{6}{36}$
& $\frac{5}{36}$
& $\frac{4}{36}$
& $\frac{3}{36}$
& $\frac{2}{36}$
& $\frac{1}{36}$ \\
\end{tabular}
\end{center}
\end{solution}
\item Show that $E(X) = 7$ and $\mathrm{var}(X) = 210/36 = 5.833$

\begin{solution}
\begin{align*}
\begin{split}
  \E[X] &= (1/36) (2) + (2/36) (3) + (3/36) (4) + (4/36) (5) + (5/36) (6) + (6/36) (7) \\
        &\quad+ (5/36) (8) + (4/36) (9) + (3/36) (10) + (2/36) (11) + (1/36) (12)
\end{split} \\
        &= 7. \\
\begin{split}
  \var[X] &= (1/36) (2-7)^2 + (2/36) (3-7)^2 + (3/36) (4-7)^2 + (4/36) (5-7)^2
           + (5/36) (6-7)^2 \\
        &\quad+ (6/36) (7-7)^2 + (5/36) (8-7)^2 + (4/36) (9-7)^2 + (3/36) (10-7)^2 \\
        &\quad\quad+ (2/36) (11-7)^2 + (1/36) (12-7)^2
\end{split} \\
        &= \frac{210}{36} \\
        &= 5.833.
\end{align*}
\end{solution}

\item The distribution of $X$ would look somewhat bell-shaped. (This is not a coincidence. The more
dice you toss, the closer the distribution of the sum comes to a normal
distribution. More on this later in the course.) For now, let's see how well
the empirical rule works. Show that the probability that $X$ is within 
$E(X) \pm \mathrm{sd}(X)$ is $24/36=0.667$.  
Show that the probability that $X$ is within $E(X) \pm 2\mathrm{sd}(X)$ is $34/36=0.944$. 
\begin{solution}
We have that
\begin{align*}
  \mu &= E[X] = 7 \\
  \sigma &= \sqrt{\var(X)} 
          = \sqrt{210/36}
          = 2.415.
\end{align*}
We can compute the $z$ scores for the different values of $x$ using the
formula $z = (x - \mu) / \sigma$.

\begin{center}
\begin{tabular}{c|ccccccccccc}
$x$ & 2 & 3 & 4 & 5 & 6 & 7 & 8 & 9 & 10 & 11 & 12 \\
\hline
$z$ & $-2.07$ & $-1.66$ & $-1.24$ & $-0.83$ & $-0.41$ & $0.00$ & $0.41$ & $0.83$ & $1.24$ & $1.66$ & $2.07$
\end{tabular}
\end{center}

With this table, we can see that
\begin{align*}
  P(-1 < Z < 1)
    &= P(X=5) + P(X=6) + P(X=7) + P(X=8) + P(X=9) \\
    &= \tfrac{4}{36} + \tfrac{5}{36} + \tfrac{6}{36} + \tfrac{5}{36} + \tfrac{4}{36} \\
    &= \tfrac{24}{36} \\
    &= 0.6667. \\
\end{align*}
Also,
\begin{align*}
  P(-2 < Z < 2)
    &= 1 - \{ P(X=2) + P(X=12) \} \\
    &= 1 - (\tfrac{1}{36} + \tfrac{1}{36}) \\
    &= \tfrac{34}{36} \\
    &= 0.9444.
\end{align*}


\end{solution}

\end{enumerate}

\end{problem}




\end{document}


