%%%%%%%%%%%%%%%%%%%%%%%%%%%%%%%%%%%%%%%%%%%%%%%
%%%This is a science homework template. Modify the preamble to suit your needs. 
%The junk text is   there for you to immediately see how the headers/footers look at first 
%typesetting.


\documentclass[answers,11pt]{exam}

\makeatletter
\newcommand{\course}[1]{\def\@course{#1}}
\makeatother

\makeatletter

\makeatletter
\pagestyle{headandfoot}
\ifprintanswers
\firstpageheader{}{\large \textbf{\@title ~-- Solutions} \\ \@course}{}
\else
\coverheader{}{\large \textbf{\@title} \\ \@course}{}
\fi
\makeatother


%AMS-TeX packages
\usepackage{amssymb,amsmath,amsthm}
%geometry (sets margin) and other useful packages
\usepackage{alltt}
\usepackage{booktabs}
\usepackage{enumerate}
\usepackage{fullpage}
\usepackage{graphicx,ctable,booktabs}
\usepackage{hyperref}


%
%Redefining sections as problems
%
\makeatletter
\newenvironment{problem}{\@startsection
       {section}
       {1}
       {-.2em}
       {-3.5ex plus -1ex minus -.2ex}
       {2.3ex plus .2ex}
       {\pagebreak[3]%forces pagebreak when space is small; use \eject for better results
       \large\bf\noindent{Problem }
       }
       }
       {%\vspace{1ex}\begin{center} \rule{0.3\linewidth}{.3pt}\end{center}}
       \begin{center}\large\bf \ldots\ldots\ldots\end{center}}
\makeatother



\DeclareMathOperator*{\sd}{sd}


%
%Fancy-header package to modify header/page numbering 
%
%\usepackage{fancyhdr}
%\pagestyle{fancy}
%\addtolength{\headwidth}{\marginparsep} %these change header-rule width
%\addtolength{\headwidth}{\marginparwidth}
%\lhead{Problem \thesection}
%\chead{} 
%\rhead{\thepage} 
%\lfoot{\small\scshape course name}
%\cfoot{} 
%\rfoot{\footnotesize HW \#1}
%\renewcommand{\headrulewidth}{.3pt}
%\renewcommand{\footrulewidth}{.3pt}
%\setlength\voffset{-0.25in}
%\setlength\textheight{648pt}

%%%%%%%%%%%%%%%%%%%%%%%%%%%%%%%%%%%%%%%%%%%%%%%

%
%Contents of problem set
%    
\begin{document}

\begin{center}
  \large
  \textbf{Homework \#2 -- Due Thursday, Jul.~19} \\
  STAT-UB.0001 -- Statistics for Business Control \\
\end{center}


\thispagestyle{empty}



\begin{problem}{}
The U.S. Census Bureau reports the percentage of mothers in the workforce who have infant children.
The following table gives a breakdown of the martial status and working status of the 1.8 million mothers with infant children in the year 2010.
(The numbers in the table are in thousands.) A mother with infant children is to be selected randomly.

\begin{table}[h!]
\begin{center}
\begin{tabular}{|c|c|c|} \hline
 & Working & Not Working \\ \hline
Married and living with husband & 1174 & 89\\ \hline
All other arrangements & 416 & 121  \\ \hline
\end{tabular}
\end{center}
\end{table}

Let A be the event ``Mom with infant works'', and let B be the event ``Mom with infant is married and living with husband''.
\begin{enumerate}[(a)]
\item Find $P(A)$, $P(B)$, and $P(A\mid B)$.
\begin{solution}
\begin{align*}
P(A) &= \frac{1174+416}{1800} = 0.883\\
P(B) &= \frac{1174+89}{1800} = 0.702\\
P(A\mid B) &= \frac{1174}{1174+89} = 0.930
\end{align*}
\end{solution}
\item Are the events A and B independent?
\begin{solution}
No, because $P(A) \ne P(A\mid B)$.
\end{solution}
\item Find $P(B\mid A)$ and $P(B \mid A^c)$. Are they equal?
\begin{solution}
\begin{align*}
P(B\mid A) &= \frac{1174}{1174+416}=0.738\\
P(B\mid A^c) &= \frac{89}{89+121}=0.424
\end{align*}
Not equal.

\end{solution}
\end{enumerate}
\end{problem}

\begin{problem}{}
The local area network (LAN) at Stern is temporarily shut down. Previous shutdowns have been due to hardware failure, software failure, or power failure. 
Maintenance engineers have determined that the probabilities of hardware, software, and power failures are 0.01, 0.05 and 0.02, respectively.
They have also determined that if the system experiences hardware problems, LAN shuts down 73\% of the time. 
Similarly, if software problems occur, LAN shuts down 12\% of the time; and if power problems occur, LAN shuts down 88\% of the time.
\begin{enumerate}[(a)]
\item Express the information given in the problem statement as probabilities or conditional probabilities. Example: P(Hardware)=0.01.
\begin{solution}
\begin{align*}
&P(\text{Hardware})=0.01, P(\text{LAN}\mid \text{Hardware})=0.73\\
&P(\text{Software})=0.05, P(\text{LAN}\mid \text{Software})=0.12\\
&P(\text{Power})=0.02, P(\text{LAN}\mid \text{Power})=0.88
\end{align*}
\end{solution}
\item What is the probability that LAN will shutdown (in the future)?
\begin{solution}
\begin{align*}
P(\text{LAN}) &= P(\text{LAN}\cap \text{Hardware}) + P(\text{LAN}\cap \text{Software}) + P(\text{LAN}\cap \text{Power}) \\
        &= P(\text{Hardware}) P(\text{LAN} \mid \text{Hardware}) \\
        &+P(\text{Software}) P(\text{LAN} \mid \text{Software}) \\
        & +P(\text{Power}) P(\text{LAN} \mid \text{Power})  \\
            &= (0.01)(0.73) + (0.05)(0.12) + (0.02)(0.88) \\
            &= 0.0309
\end{align*}
\end{solution}
\item Given the fact that LAN is shut down, what is the probability that the current shutdown is due to hardware failure? Software failure? Power failure?
\begin{solution}
\begin{align*}
P(\text{Hardware}\mid \text{LAN}) &= \frac{P(\text{LAN}\cap \text{Hardware})}{P(\text{LAN})}=\frac{(0.01)(0.73)}{0.0309}=0.236\\
P(\text{Software}\mid \text{LAN}) &= \frac{P(\text{LAN}\cap \text{Software})}{P(\text{LAN})}=\frac{(0.05)(0.12)}{0.0309}=0.194\\
P(\text{Power}\mid \text{LAN}) &= \frac{P(\text{LAN}\cap \text{Power})}{P(\text{LAN})}=\frac{(0.02)(0.88)}{0.0309}=0.570
\end{align*}
\end{solution}
\end{enumerate}
\end{problem}

\begin{problem}{}
\begin{enumerate}[(a)]
\item There are 3 male and 3 female students sitting in a row, and they chose their seats randomly. What is the probably of the event "No two males sit together and no two females sit together"?
(Hint: The gender of the seated students can only be FMFMFM or MFMFMF.)
\begin{solution}
There are a total of $6!$ ways to arrange the 6 students. If gender is FMFMFM, there are $3!3!=36$ ways to arrange them;
similarly if the gender is MFMFMF, there are $3!3!=36$ ways as well. Therefore the probability is 
$$ \frac{3!3!+3!3!}{6!}=0.1.$$
\end{solution}
\item Compute the probability of randomly drawing five cards from a deck, and getting two Aces. (Hint: A deck has 52 cards, and 4 of them are Aces.) 
\begin{solution}
The probability is 
\begin{align*}
&\frac{\#\{\text{select 2 aces out of 4}\}\#\{\text{select 3 cards out of 48 non-aces}\}}
{\#\{\text{select 5 out of 52}\}} \\
&=\frac{\binom{4}{2} \binom{48}{3}}{\binom{52}{5}} = 0.0399
\end{align*}
\end{solution}
\end{enumerate}
\end{problem}


\begin{problem}{}
Consider the probability distribution function for the random variable $X$ shown here:
\begin{center}
\begin{tabular}{|c|c|c|c|c|c|c|} \hline
x & 10& 20 &30 &40 &50 &60 \\ \hline
$p(x)$ & 0.05&0.20 &0.30 & 0.25&0.10 &0.10 \\ \hline
\end{tabular}
\end{center}
\begin{enumerate}[(a)]
\item Find $\mu$, $\sigma^2$, and $\sigma$. Write down the formulas showing how do you get these numbers.
\begin{solution}
\begin{align*}
\mu&=(0.05)(10)+(0.20)(20) + (0.30)(30) + (0.25)(40)+(0.10)(50)+(0.10)(60)=34.5\\
\sigma^2&=(0.05)(10-34.5)^2+(0.20)(20-34.5)^2 + (0.30)(30-34.5)^2 \\
&+ (0.25)(40-34.5)^2+(0.10)(50-34.5)^2+(0.10)(60-34.5)^2=174.75\\
\sigma&=\sqrt{174.75}=13.22
\end{align*}
\end{solution}
\item Use empirical rules with $\mu$ and $\sigma$ to find the approximate 95\% interval of the probability.
\begin{solution}
By empirical rule, the approximate 95\% interval of the probability is 
$$ (\mu \pm 2\sigma) = (34.5\pm 2(13.22)) = (8.06,60.94)
$$
\end{solution}
\item What is the probability that $X$ will fall within the interval you just computed in (b)? (Hint: Use the PDF table.)
\begin{solution}
The true probability that $X$ falls within $(8.06,60.94)$ is 100\%.
\end{solution}

\end{enumerate}
\end{problem}


\end{document}


