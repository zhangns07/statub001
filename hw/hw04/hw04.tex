%%%%%%%%%%%%%%%%%%%%%%%%%%%%%%%%%%%%%%%%%%%%%%%
%%%This is a science homework template. Modify the preamble to suit your needs. 
%The junk text is   there for you to immediately see how the headers/footers look at first 
%typesetting.


\documentclass[11pt]{exam}

%AMS-TeX packages
\usepackage{amssymb,amsmath,amsthm}
%geometry (sets margin) and other useful packages
\usepackage{alltt}
\usepackage{booktabs}
\usepackage{enumerate}
\usepackage{fullpage}
\usepackage{graphicx,ctable,booktabs}
\usepackage{hyperref}


%
%Redefining sections as problems
%
\makeatletter
\newenvironment{problem}{\@startsection
       {section}
       {1}
       {-.2em}
       {-3.5ex plus -1ex minus -.2ex}
       {2.3ex plus .2ex}
       {\pagebreak[3]%forces pagebreak when space is small; use \eject for better results
       \large\bf\noindent{Problem }
       }
       }
       {%\vspace{1ex}\begin{center} \rule{0.3\linewidth}{.3pt}\end{center}}
       \begin{center}\large\bf \ldots\ldots\ldots\end{center}}
\makeatother



\DeclareMathOperator*{\sd}{sd}


%
%Fancy-header package to modify header/page numbering 
%
%\usepackage{fancyhdr}
%\pagestyle{fancy}
%\addtolength{\headwidth}{\marginparsep} %these change header-rule width
%\addtolength{\headwidth}{\marginparwidth}
%\lhead{Problem \thesection}
%\chead{} 
%\rhead{\thepage} 
%\lfoot{\small\scshape course name}
%\cfoot{} 
%\rfoot{\footnotesize HW \#1}
%\renewcommand{\headrulewidth}{.3pt}
%\renewcommand{\footrulewidth}{.3pt}
%\setlength\voffset{-0.25in}
%\setlength\textheight{648pt}

%%%%%%%%%%%%%%%%%%%%%%%%%%%%%%%%%%%%%%%%%%%%%%%

%
%Contents of problem set
%    
\begin{document}

\begin{center}
  \large
  \textbf{Homework \#4 -- Due Thursday, Aug.~2} \\
  STAT-UB.0001 -- Statistics for Business Control \\
\end{center}


\thispagestyle{empty}


\begin{problem}{}

Find the probability that a standard normal random variable is:

\begin{enumerate}[(a)]

  \item Greater than zero

\begin{solution}
  \[
    P(Z > 0) = 0.5
  \]
\end{solution}

  \item Greater than $-1.4$

\begin{solution}
  \[
    %P(Z > -1.5) = 0.93319
  \]
\end{solution}

  \item Less than $-0.7$

\begin{solution}
  \[
   % P(Z < -0.3) = 0.3821
  \]
\end{solution}

  \item Between $-1$ and $2$

\begin{solution}
  \[
    %P(-2 \leq Z \leq 1) = 0.8413 - 0.02275 = 0.81855
  \]
\end{solution}

  \item Less than $-0.5$ or greater than $1.7$
\begin{solution}
\end{solution}

  \item Equal to $1$.

\begin{solution}
  \[
    %P(Z = 1) = 0
  \]
\end{solution}


\end{enumerate}

\end{problem}



\begin{problem}{}

Find a value of a standard normal random variable $Z$ (call it $z_0$)
such that

\begin{enumerate}[(a)]

  \item $P(Z < z_0) = .30$
\begin{solution}
  \[
    z_0 = -0.8416
  \]
\end{solution}

  \item $P(Z > z_0) = .16$

\begin{solution}
  \[
    z_0 = 1.96
  \]
\end{solution}

  \item $P(-z_0 < Z < z_0) = .90$

\begin{solution}
  \[
    z_0 = 1.4051
  \]
\end{solution}

%  \item $P(-z_0 < Z < 0) = .4798$

%  \item $P(-1 < Z < z_0) = .5328$.

\end{enumerate}

\end{problem}



\begin{problem}{}

Suppose that $X$ is normally distributed with mean $11$ and standard
deviation $2$. Find

\begin{enumerate}[(a)]

  \item $P(10 < X < 12)$

\begin{solution}
\begin{align*}
  P(10 < X < 12)
    &= P\left(\frac{10-11}{2} < \frac{X - 11}{2} < \frac{12 - 11}{2}\right) \\
    &= P(-0.5 < Z < 0.5) \\
    &= 0.3829 
\end{align*}

\end{solution}
%  \item $P(6 < X < 10)$

  \item $P(X > 7.6)$.

\begin{solution}
\begin{align*}
  P(X > 7.6)
    &= P\left(\frac{X - 11}{2} > \frac{7.6 - 11}{2}\right) \\
    &= P(Z > -1.7) \\
    &= 1 - 0.04457 \\
    &= 0.95543.
\end{align*}
\end{solution}

\end{enumerate}

\end{problem}



\begin{problem}{}

A Pepsi machine in a Burger King store can be regulated so that it dispenses
an average of $\mu$ ounces per cup. If the amount dispensed is normally
distributed with standard deviation $0.3$ ounces, what should be the setting
for $\mu$ so that $8$-ounce cups will overflow only $1\%$ of the time?

\begin{solution}
Let $X$ be the amount dispensed from the machine.  We want to find $\mu$ such
that $P(X > 8) = 0.01$.  Thus,
\begin{align*}
  P\left(\frac{X - \mu}{0.2} > \frac{8 - \mu}{0.2}\right) &= 0.01 \\
  P\left(Z > \frac{8 - \mu}{0.2}\right) &= 0.01,
\end{align*}
so that
\[
  \frac{8 - \mu}{0.2} = 2.3263,
\]
and hence
\begin{align*}
 \mu &= 8 - (0.2) (2.3263) \\
     &= 7.53474.
\end{align*}
\end{solution}

\end{problem}



\begin{problem}{}

Suppose that annual stock returns for a particular company are normally
distributed with a mean of $18\%$ and a standard deviation of $6\%$. You are
going to invest in this stock for one year.  (Note: In reality, annual returns
tend to be more nearly normally distributed than daily returns.)
Find that the probability that your one-year return will exceed $26\%$.

\begin{solution}
Let $X$ be the annual return, in percent.  This is a normal random variable
with mean $\mu = 16$ and standard deviation $\sigma = 10$.  The probability of
interest is
\begin{align*}
  P(X > 30)
    &= P\left(\frac{X - \mu}{\sigma} > \frac{30 - \mu}{\sigma}\right) \\
    &= P\left(Z > \frac{30 - 16}{10}\right) \\
    &= P(Z > 1.4) \\
    &= 0.08076.
\end{align*}
\end{solution}

%  \item Find that probability that your one-year return will be negative.

\end{problem}



\begin{problem}{}

If the population standard deviation is $2.5$ and we take a random sample of
size $81$, what is $\sd (\bar X)$?   Note: this quantity is known as
the ``standard error of the mean.'' 

\begin{solution}
\[
  \sd(\bar X) = \frac{\sigma}{\sqrt{n}} = \frac{2.3}{\sqrt{64}} = 0.2875.
\]

\end{solution}
\end{problem}


\begin{problem}{}

Suppose that daily returns on a portfolio are independent, with a mean of
$0.03\%$ and a standard deviation of $1\%$.  Approximately what is the
probability that the average daily return over the next 100 days will be
between $0.2\%$ and $0.3\%$?

\begin{solution}

Let $\bar X$ denote the average return over the next 100 days, in percent.
Then, by the central limit theorem, $\bar X$ is approximately normal with mean
and standard deviation
\begin{align*}
  \mu_{\bar X} &= \mu = 0.03, \\
  \sigma_{\bar X} &= \frac{\sigma}{\sqrt{n}} = \frac{1}{\sqrt{100}} = 0.1.
\end{align*}
The probability of interest is
\begin{align*}
  P(0.2 < X < 0.3)
    &= P\left(\frac{0.2 - 0.03}{0.1}
          < \frac{X - \mu_{\bar X}}{\sigma_{\bar X}}
          < \frac{0.3 - 0.03}{0.1}\right) \\
    &= P(1.7 < Z < 2.7) \\
    &= 0.996533 - 0.95543 \\
    &= 0.041103.
\end{align*}
\end{solution}

\end{problem}



\begin{problem}{}

If we throw $n$ dice where $n$ is large, why can we think of the distribution
of the sum as being approximately normal?

\begin{solution}
The sum is equal to $n \bar X$, where $\bar X$ is the average value of the $n$
rules.  By the central limit theorem, $\bar X$ is approximately normal if $n$
is large.  Further, if we scale a normal random variable by a constant ($n$),
then we get a normal random variable.  Thus, $n \bar X$, the sum, is
approximately normal.
\end{solution}

\end{problem}



\begin{problem}{}

Suppose that an auto factory has $8$ assembly lines, operating
independently.  For each assembly line, the number of autos produced per day
has a normal distribution with mean of $20$ and a standard deviation of $4$. 
Approximately what is the
probability that $120$ or fewer autos will be produced tomorrow?

\begin{solution}
Let $\bar X$ be the average number produced by the $10$ assembly lines.  Then,
by the central limit theorem, $\bar X$ is approximately normal with mean and
standard deviation
\begin{align*}
  \mu_{\bar X} &= 20, \\
  \sigma_{\bar X} &= \frac{3}{\sqrt{10}} = 0.9487.
\end{align*}
To produce a total of $180$ or fewer autos, the $10$ factories must produce an
average of $180/10 = 18$ or fewer.  The probability of interest is
\begin{align*}
  P(\bar X < 18)
    &= P\left(\frac{\bar X - \mu_{\bar X}}{\sigma_{\bar X}}
          < \frac{18 - 20}{0.9487}\right) \\
    &\approx P(Z < -2.1) \\
    &= 0.01786.
\end{align*}
\end{solution}

\end{problem}

\end{document}


