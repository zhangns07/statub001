\documentclass{beamer}

\usetheme{default}
\usecolortheme{rose}
\usepackage{hyperref}
\setbeamerfont{alerted text}{series=\itshape}
\newcommand{\ignore}[1]{}

\title{Populations and Samples}

% A subtitle is optional and this may be deleted
\subtitle{STAT-UB.0001 Statistics for Business Control}

\author{Ningshan Zhang}
% - Give the names in the same order as the appear in the paper.
% - Use the \inst{?} command only if the authors have different
%   affiliation.

\institute[New York University] % (optional, but mostly needed)
{
  IOMS Department\\
  nzhang@stern.nyu.edu
}
\date{July 3, 2018}
\AtBeginSubsection[]
{
  \begin{frame}<beamer>{Outline}
    \tableofcontents[currentsection,currentsubsection]
  \end{frame}
}

% Let's get started
\begin{document}

%-------------------
\begin{frame}
  \titlepage
\end{frame}



% Section and subsections will appear in the presentation overview
% and table of contents.
\section{Course Logistics}

%-------------------
\begin{frame}{Course Logistics}
See the syllabus on NYU Classes.
\end{frame}

\section{Introduction to Statistics}

%-------------------
\begin{frame}{Introduction to Statistics}
The word 'Statistics' means
\begin{itemize}
\item In ordinary conversations, ``statistics'' means a collection of numbers. (Sport statistics, census statistics, etc. )
\item In this course, ``statistics'' is an analytical discipline. 
\end{itemize}
\vspace{\stretch{0.5}}

Wikipedia: Statistics is a branch of mathematics dealing with the collection, analysis, interpretation, presentation, and organization of data. 
\vspace{\stretch{0.5}}

``Statistics is using a \alert{sample} to make a statement about a \alert{population}.''
\end{frame}

%-------------------
\begin{frame}{Introduction to Statistics}
Why statistics? They say...\footnote{\url{https://youtu.be/d5urHxg_KSs}}
\begin{itemize}
\item Apply to various fields.
\item Use math to solve problems.
\item Not a major, but a set of skills.
\item Open so many opportunities.
\end{itemize}
\end{frame}


%-------------------
\begin{frame}{Introduction to Statistics}
Statistics is used in a variety of fields:
\begin{itemize}
\item Finance (Banking, Risk Management)
\item Marketing (Targeted Advertising)
\item Health (Drug Development)
\item Politics (Election Polls) 
\item Education (College Admissions Data, Teacher Evaluations) 
\item Many more...
\end{itemize}
\end{frame}

\begin{frame}{Course Topics}
\begin{itemize}
\item Populations and Samples
\item Descriptive Statistics
\item Probability
\begin{itemize}
\item Basic Probability
\item Conditional Probability
\item Discrete Random Variables
\item Continuous Random Variables
\end{itemize}
\item Inferential Statistics
\begin{itemize}
\item Sampling Distributions
\item Confidence Intervals
\item Hypothesis Tests
\end{itemize}
\end{itemize}
\end{frame}

%-------------------
\begin{frame}{Populations vs Samples}
``Statistics is using a \alert{sample} to make a statement about a \alert{population}.''
\vspace{\stretch{0.5}}

\begin{block}{ Population}
 The set of items or individuals that we are interested in studying and drawing conclusions about.
 \begin{itemize}
 \item What we care about. The world. 
 \end{itemize}
\end{block}
\begin{block}{ Sample}
 A subset of items or individuals from the population, i.e. 
 the data.
  \begin{itemize}
 \item What we have access to. A measurement of the world. 
 \end{itemize}
\end{block}
\end{frame}

%-------------------
\begin{frame}{Populations vs Samples}
\begin{block}{Example}
You want to estimate the average amount of time that NYU undergraduates spend on social networking websites per day. % based on data collected from our class.
\end{block}
\begin{itemize}
\item What is the population?
\vspace{\stretch{0.5}}
\item How to collect a sample?
\end{itemize}
\end{frame}


%-------------------
\begin{frame}{Representative}
A representative sample is a sample whose properties accurately reflect those of the population. 
\begin{itemize}
\item Not possible to guarantee a representative sample in practice.
\item If we know the population, take the sample to be equal to the population, which is representative.
\item Otherwise, take an \alert{unbiased} sample.
\end{itemize}
\end{frame}


%-------------------
\begin{frame}{Unbiased sample}
A sample is unbiased when every member of the population has an equal chance of being included in the sample.
\begin{itemize}
\item An unbiased sample is  not guaranteed to be representative, but  it will be representative on average.
\item A large unbiased sample will be approximately representative. 
\end{itemize}
\vspace{\stretch{0.5}}

If some member of the population is more likely to be in the sample, then the sample is biased. 

\end{frame}


%-------------------
\begin{frame}{Sources of Bias}
\begin{block}{Selection Bias}
There is a systematic tendency for one group to be overrepresented or underrepresented. In other words, some type of individuals are more (or less) likely to be surveyed. 
\end{block}

Example: The Wall Street Journal wants to estimate its popularity amongst college students.  To do this, an employee surveys a random sample of Stern undergraduates.
\vspace{\stretch{0.5}}

Any problem?
\end{frame}

%-------------------
\begin{frame}{Sources of Bias}
\begin{block}{Nonresponse Bias}
The researcher is unable to obtain data on all experimental units selected for the sample. In other words, some type of individuals are more (or less) likely to respond. 
\end{block}
Example: A university uses course faculty evaluations to assess the quality of its instructors.  There is no consequence if a student does not participate.

\vspace{\stretch{0.5}}
Any problem?
\end{frame}

%-------------------
\begin{frame}{Sources of Bias}
Example: During WWII, statistician Abraham Wald\footnote{Known for the Wald test.} was asked to help the British decide where to add armor to their bombers. Wald recommended adding more armor to the places where there was no damage! 

\vspace{\stretch{0.5}}
Why?
\end{frame}


%-------------------
\begin{frame}{Summary}
\begin{itemize}
\item Populations and samples
\item Representative sample
\item Unbiased sample
\item Sources of bias
\end{itemize}

\end{frame}



\end{document}


