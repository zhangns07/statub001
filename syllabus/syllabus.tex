\documentclass[11pt]{article}

\usepackage{array}
\usepackage{booktabs}
\usepackage{fullpage}
\usepackage{supertabular}
\usepackage{hyperref} % must be last usepackage

\newcommand{\PreserveBackslash}[1]{\let\temp=\\#1\let\\=\temp}
\let\PBS=\PreserveBackslash


\begin{document}

\begin{center}
  \large
  \textbf{Syllabus -- Summer 2018} \\
  
STAT-UB1-001 -- Statistics for Business Control \\
\end{center}


\section*{Meeting Time \& Place}
Lectures: Tuesdays and Thursdays, 9:00 AM -- 12:10 PM \\
Midterm Exam (Tentative): Thursday, July 19, 9:00 AM -- 12:10 PM \\
Final Exam: Thursday, August 9, 9:00 AM -- 12:10 AM \\
Class Room: Tisch UC19


\section*{Course Staff}

\begin{tabbing}
  MMMMMMMMMMMM\= \kill
  Instructor:   \> Ningshan Zhang \\
  E-mail:       \> nzhang@stern.nyu.edu \\
  Office:       \> KMC (Kaufman Management Center) 8-174 \\
  Office Hours: \> Wednesday and Friday, 10:00 AM -- 11:30 AM \\
\end{tabbing}

\section*{Course Objectives}

The basic objective of this course is to provide the business student with a
strong fundamental understanding of statistics and its applications. Students
will learn statistical applications utilizing real world examples and exercises 
from various fields. Topics include the control of statistical bias, data 
presentation, appreciation of probability and randomness, random variables, and
statistical inference. More specifically, chapter 1-8 on textbook.

\section*{Course Organization}
You will be responsible for the material contained in course lectures,
handouts, and homework assignments. Class consists of lecture and handouts
which will be distributed at the beginning of each class. Class material will 
also be posted on NYU Classes. 

We will use Minitab, Release 15/16, for data analysis. A student version of
this software is included with the textbook by the NYU bookstore. 
The differences between Minitab 15/16 and the student version are minor.
Minitab 16 is available in the computer labs and online at
\url{http://apps.stern.nyu.edu}. More instructions for using Minitab 16 can be found
on NYU Class/Resources/Minitab. 

\section*{Texts and Materials}

\begin{enumerate}

    \item McClave, Benson, and Sincich, Statistics for Business and Economics,
  Third Custom Edition for New York University, Pearson.  
    \item Minitab 15/16 or Minitab student version.  \emph{Minitab will only run on a Windows PC.  If you do not have a
  Windows PC, then you can run Minitab in the student computer labs or
  online via }\url{http://apps.stern.nyu.edu}.

\end{enumerate}

\section*{Grading Policy}

We will have homework, one midterm, and one final exam. Your grade
will be based on these, as well as class participation.

\begin{tabbing}
  MMMMMMMM\= \kill
Participation \> 20\% \\
  Homework  \> 20\% \\
  Midterm   \> 25\% \\
  Final     \> 35\%
\end{tabbing}


\section*{Exams}

There are one midterm and one final exam.  If you have a potential conflict
during one of the exams, you must discuss the matter with the instructor
during the first week of class.  You should check the final exam
schedule before you buy any plane tickets or make any other travel plans.

If you miss an exam, you will not be given the opportunity to make up the test
except in cases of a documented medical illness.  You must have a signed note
from a doctor.  In extreme cases when you miss an exam due to documented
medical illness, at the instructor's discretion you may have the opportunity
to take a make-up exam.  This exam will not necessarily have the same format
as the original exam (it could be an oral exam, for example).

\section*{Homework}
We will have one homework per week. It will be posted on Thursday, and it will be due at the beginning of next Tuesday's class. 
Homework counts for 20\% of your grade.  
Students are expected to come to class prepared having read text and assigned
readings prior to class. It is
suggested that students keep a copy of their homework to study from (in case
it is not returned before an exam). 


\section*{Late Policy}

Assignments are due at the beginning of class on the day specified, and late
assignments are strongly discouraged.  That said, there are unforeseen
emergencies (illness, etc.) that cannot always be planned for in advance. 
Instead of having to ask for special allowances on an individual basis, I
give each of you the privilege of dropping \textbf{one} assignment in case of crisis. 

Although this policy is not intended to cover poor planning or procrastination,
I won't ask for justification and will assume you will use your privilege  fairly and wisely.  


\section*{Regrading}

If you find what you believe to be a grading error on an assignment or exam,
you must bring the matter to the attention of the course staff no later than 2
days after the assignment was handed back.  \emph{Requests for grading
adjustments after this will not be considered.}  This includes cases when the
written grade does not match the recorded grade on the course website.
Discuss homework grading issues with the teaching fellow, and discuss midterm
grading issues with the instructor.  You must discuss all grading issues in
person, during office hours.

If you erase anything, change any answers, or add any notes after your
assignment or exam has been graded, you may not submit the assignment or exam
for regrading.  If you modify an assignment or exam in any way after it gets
returned, and then you submit that assignment or exam for regrading, this will
be considered to be a violation of the academic integrity policy.



\section*{Class Attendance And Participation}
Participation is an essential part of learning in this course.  Students are
expected to participate in all facets of classroom learning. I expect you to
take an active role in learning Statistics. I may call on you, and I want you
to ask questions. There's no such thing as a ``bad'' question or comment, so
don't be afraid to speak up (in an orderly fashion). This helps me to identify
points that I need to explain further. Participation counts for 20\% of your grade.

\section*{Classroom Norms}

Cell phones, smartphones and similar electronic devices are a disturbance to
both students and professors.  All such electronic devices must be turned off
prior to the start of each class meeting.  Laptops are not permitted in class.


\section*{Ethical Guidelines: Student Code of Conduct}

All students are expected to follow the Stern Code of Conduct
\url{http://www.stern.nyu.edu/uc/codeofconduct}. A student’s responsibilities
include, but are not limited to, the following:

\begin{itemize}

  \item A duty to acknowledge the work and efforts of others when submitting
  work as one’s own.  Ideas, data, direct quotations, paraphrasing, creative
  expression, or any other incorporation of the work of others must be clearly
  referenced.

  \item A duty to exercise the utmost integrity when preparing for and
  completing examinations, including an obligation to report any observed
  violations.

\end{itemize}

\noindent To minimize the temptation for copying or sharing during
an exam, there will be multiple versions of every exam, and the seating order
will be randomly assigned.


\section*{Students with Disabilities}

Students whose class performance may be affected due to a disability should
notify the professor immediately so that arrangements can be made in
consultation with the Henry and Lucy Moses Center for Students with
Disabilities (\url{http://www.nyu.edu/csd/}) to accommodate their needs.


\newpage

\section*{Tentative Schedule}

\begin{center}
%\footnotesize
\small
\renewcommand{\arraystretch}{0.78}

\tablefirsthead{
  \toprule
  \textbf{Date}
  & \textbf{Topics}
  & \textbf{Textbook Chapters}
  \\ 
  \midrule
}

\tablelasttail{
  \bottomrule
}

\begin{supertabular}
{l
>{\PBS\raggedright\hspace{0pt}}p{.50\textwidth}
>{\PBS\raggedright\hspace{0pt}\parskip=5pt}p{.15\textwidth}
%>{\PBS\raggedright\hspace{0pt}\parskip=5pt}p{.05\textwidth}
}


7/3 
& Populations and Samples, Descriptive Statistics
& 1,2
\\\\

7/5 
& Probability, Conditional Probability 1
& 3.1 - 3.5
\\\\

7/10
& Conditional Probability 2
& 3.5 - 3.7
\\\\

7/12 
&  Random Variables, Models for Counts 1
& 4.1 - 4.3
\\\\ 

7/17
& Models for Counts 2, Review for Midterm
& 4.4
\\\\

7/19
& \textbf{Midterm}
&
\\\\

7/24
& The Normal Model
& 4.5 - 4.7
\\\\
7/26
& The Central Limit Theorem, Confidence Intervals 
& 5, 6
\\\\
7/31
& Statistical Tests  
& 7
\\\\

8/2 
& Comparison
& 8
\\\\ 

8/7 
& Review
& 
\\\\ 

8/9
& \textbf{Final Exam} 
&
\\
\end{supertabular}
\end{center}




\end{document}
